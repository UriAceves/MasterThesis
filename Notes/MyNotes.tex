%%%%%%%%%%%%%%%%%%%%%%%%%%%%%%%%%%%%%%%%%
% Simple Sectioned Essay Template
% LaTeX Template
%
% This template has been downloaded from:
% http://www.latextemplates.com
%
% Note:
% The \lipsum[#] commands throughout this template generate dummy text
% to fill the template out. These commands should all be removed when
% writing essay content.
%
%%%%%%%%%%%%%%%%%%%%%%%%%%%%%%%%%%%%%%%%%

%----------------------------------------------------------------------------------------
%	PACKAGES AND OTHER DOCUMENT CONFIGURATIONS
%----------------------------------------------------------------------------------------

\documentclass[11pt,twoside,a4paper]{article} % Default font size is 12pt, it can be changed here

\usepackage{geometry} % Required to change the page size to A4
\geometry{legalpaper, margin=2.5cm}


\usepackage{graphicx} % Required for including pictures
\usepackage{cancel}
\usepackage{float} % Allows putting an [H] in \begin{figure} to specify the exact location of the figure
\usepackage{wrapfig} % Allows in-line images such as the example fish picture
\usepackage[english]{babel}
\usepackage[utf8]{inputenc}
\usepackage{hyperref}
\usepackage{lipsum} % Used for inserting dummy 'Lorem ipsum' text into the template}
\usepackage{bbold}
\usepackage{amssymb}
\usepackage{amsmath}
\usepackage{fancyhdr}
\usepackage{braket}
\usepackage{xcolor}
\usepackage{tikz}
\usepackage{bm}
\usepackage{tgtermes}
\usepackage{mathrsfs}

\pagestyle{fancy}
\fancyhead[RO,LE]{\thepage}
\fancyhead[LO]{MAJORANA MODES IN MATERIALS}
\fancyhead[RE]{\leftmark}

\linespread{1.0} % Line spacing

\newcommand{\r}{\bm{r}}
\newcommand{\R}{\bm{R}}

\newcommand{\vect}[1]{\boldsymbol{#1}} % Uncomment for BOLD vectors.
%\newcommand{\vect}[1]{\vec{#1}} % Uncomment for ARROW vectors.

%\setlength\parindent{0pt} % Uncomment to remove all indentation from paragraphs

\graphicspath{{Pictures/}} % Specifies the directory where pictures are stored
\newcommand{\norm}[1]{\left\|#1\right\|}

% \renewcommand{\thesection}{\Roman{section}}
% \renewcommand{\thesubsection}{\thesection.\Roman{subsection} }

\begin{document}

%----------------------------------------------------------------------------------------
%	TITLE PAGE
%----------------------------------------------------------------------------------------

\begin{titlepage}
\newcommand{\HRule}{\rule{\linewidth}{0.5mm}} % Defines a new command for the horizontal lines, change thickness here

\center % Center everything on the page
\textsc{\LARGE Forschungszentrum Jülich\\\vspace{1cm}PGI- 1, IAS- 1}\\[5cm] % Name of your university/college

\HRule \\[0.4cm]
{ \huge \bfseries Majorana Modes in Materials}\\[0.4cm] % Title of your document
\HRule \\[1cm]

\begin{minipage}{0.4\textwidth}
\begin{flushleft} \large
\emph{Author:}\\
Uriel \textsc{Aceves} % Your name
\end{flushleft}
\end{minipage}
~
\begin{minipage}{0.4\textwidth}
\begin{flushright} \large
\emph{In collaboration with:} \\
Juba \textsc{Bouaziz} \\% Supervisor's Name
Samir \textsc{Lounis}\\ % Supervisor's Name\\
Philipp \textsc{Rüßmann} % Supervisor's Name
\end{flushright}
\end{minipage}\\[3cm]

{\large \today}\\[5cm] % Date, change the \today to a set date if you want to be precise

\includegraphics[width=0.5\textwidth]{logo.jpeg}\\[1cm] % Include a department/university logo - this will require the graphicx package

\vfill % Fill the rest of the page with whitespace

\end{titlepage}

%----------------------------------------------------------------------------------------
%	TABLE OF CONTENTS
%----------------------------------------------------------------------------------------

\tableofcontents % Include a table of contents

\newpage % Begins the essay on a new page instead of on the same page as the table of contents

%----------------------------------------------------------------------------------------
%	INTRODUCTION
%----------------------------------------------------------------------------------------


\section*{Disclaimer}

This notes are my own revision of literature concerning this topic. The main source for every section is always mentioned at the first paragraph. Explicit calculations here are made with the idea of clarifying some missing steps on the papers. The sections are chronologically organized (as I read them), so the sequence might not be the optimal to learn about this. Eventually I will organize it in a nice way to take someone from zero to hero.\\

This is a color code I will use to signal work that I need to do and to determine what to address first, it would be useful to add the date on each box.

\begin{itemize}
    \item \colorbox{green}{dd.mm.yy Life can go on if this is ignored.}
    \item \colorbox{yellow}{dd.mm.yy Not urgent, but needs to be done at some point.}
    \item \colorbox{orange}{dd.mm.yy Second in the list of importance.}
    \item \colorbox{red}{dd.mm.yy Should take care of this things first.}
\end{itemize}

\section{Abstract}

\section{Derivation of the Dirac equation}\label{dirEq}
\subsection{Brief introduction to second quantization}

The first action we need to take is quickly clarify the language we will be using here, the language of second quantization. The mathematical objects we are going to manipulate are quantum states, and consider we have an abstract space full with them, all possible states. For simplicity, consider a system of sites each of one can be either empty or occupied by \textbf{only one} particle. At the beginning this abstractions can be hard to catch, so we will check a small example. Assume we only have three sites, for example three small boxes, these boxes have enough capacity to contain a tennis ball inside them, but no more than one. The states of our abstract space of states are given then by the different combinations of empty and occupied boxes, for example; if the first box is occupied and the rest empty we will represent it as
\begin{equation*}
    \ket{1}_1\bigotimes\ket{0}_2\bigotimes\ket{0}_3 = \ket{100},
\end{equation*}
where the subindex in the lefthand side of the equation labels the site or box. The righthand side just shows a condensed representation of the same object, where the indices are left out. Now it must be easier to think and list all possible states of our example. Once we are clear in which space we are playing a sensible question might arise, and that is, how can I go from one state to a different one? For example how can we go from $\ket{000}$ to $\ket{100}$. We can stop to think for a moment here, what would it mean to go from $\ket{000}$ to $\ket{100}$? Well, $\ket{000}$ is a state in which all sites are empty, whereas in $\ket{100}$ the first site is occupied and the rest are empty, hence, the obvious difference between both states is that in the second, one particle is occupying site 1 (translated to the box example a ball is now stored in box 1). It would be reasonable then, to say that in order to go from $\ket{000}$ to $\ket{100}$ we would need somehow to place or create a particle in site 1. We can indicate that action using an operator that will go by the name of  $c_1^\dagger$, therefore
\begin{equation*}
    c_1^\dagger \ket{000} = \ket{100}.
\end{equation*}
The inverse process of going from $\ket{100}$ to $\ket{000}$, should now be easy to elucidate, what we need to do is to destroy or remove the particle at site 1, the operator that will indicate this action will be called $c_1$, and acts as follows
\begin{equation*}
    c_1 \ket{100} = \ket{000}.
\end{equation*}
By know the reader might be suspecting that this operations are not exclusive for site 1, and similar operators should exist for the others. We can take our curiosity further and ask, what if I want to destroy a particle at site 1 but there is none there? e.g. $c_1\ket{000}$. Or what if I try to create one particle where there is one already?\footnote{Remember that at the beginning we specified that each site can be occupied by just one particle.} e.g $c_1^\dagger\ket{100}$. For us, both actions will be forbidden, and we will enforce the rules by imposing that the result to both operations will be zero i.e.
\begin{eqnarray}
    c_1\ket{000} = 0,\nonumber\\
    c_1^\dagger\ket{100} = 0.\nonumber
\end{eqnarray}
Let's not get confused between 0 the number and the state with all empty sites $\ket{000}$, the number $0$ means we don't have a system anymore, we lost it, and there is nothing left.\\

Now we can extend this notion to an abstract space of n sites; the elements of it are given by $\ket{n_1,n_2,\ldots,n_n}$ where $n_i$ is the number of times state $i$ is occupied, in our example this is the number of particles present in site $i$. In this space, the operators we discussed previously would act as follows 
\begin{eqnarray}
    c_i \ket{n_1,\ldots,n_i,\ldots,n_n} = n_i\ket{n_1,\ldots,0_i,\ldots,n_n},\label{def1.1}\\
    c_i^\dagger \ket{n_1,\ldots,n_i,\ldots,n_n} = (1-n_i)\ket{n_1,\ldots,1_i,\ldots,n_n}.\label{def1.2}
\end{eqnarray}
So far, this definitions have been no more than a source of happiness for us, buy they are not exempt of problems as we will see next. Consider a system of fermionic sites, assume we have two fermions, and we tag them with colours, one \textcolor{blue}{blue} and one \textcolor{red}{red}\footnote{Fermions should be indistinguishable, but let's continue like this for the sake of the argument.} just to track changes easily (the tag nevertheless has no physical meaning). Our goal is to place them in our three sites toy system, one in site 1, and one in site 3, it doesn't matter which goes where. We will create the particles in order, first the red one, then blue. For this case we have two possibilities, placing the first one in site 1 and the second one in site 3, or the other way around, in our new language this is represented by

\begin{equation}
    \begin{aligned}[b]
      \textcolor{blue}{c_3^\dagger}\textcolor{red}{c_1^\dagger}\ket{000} &= \textcolor{blue}{c_3^\dagger}\textcolor{red}{(1-n_1)}\ket{\textcolor{red}{1}00} \\&= \textcolor{blue}{c_3^\dagger}\textcolor{red}{(1-0)}\ket{\textcolor{red}{1}00} \\&=  \textcolor{blue}{c_3^\dagger}\ket{\textcolor{red}{1}00}\\&= \textcolor{blue}{(1-n_3)}\ket{\textcolor{red}{1}0\textcolor{blue}{1}} \\&= \textcolor{blue}{(1-0)}\ket{\textcolor{red}{1}0\textcolor{blue}{1}}\\&=\ket{\textcolor{red}{1}0\textcolor{blue}{1}},\label{wrong1}
    \end{aligned}
\end{equation}

\begin{equation}
    \begin{aligned}[b]
      \textcolor{blue}{c_1^\dagger}\textcolor{red}{c_3^\dagger}\ket{000} &= \textcolor{blue}{c_1^\dagger} \textcolor{red}{(1-n_3)} \ket{00\textcolor{red}{1}} \\&= \textcolor{blue}{c_1^\dagger} \textcolor{red}{(1-0)} \ket{00\textcolor{red}{1}} \\&= \textcolor{blue}{c_1^\dagger} \ket{00\textcolor{red}{1}} \\&= \textcolor{blue}{(1-n_1)} \ket{\textcolor{blue}{1}0\textcolor{red}{1}}\\&= \textcolor{blue}{(1-0)} \ket{\textcolor{blue}{1}0\textcolor{red}{1}}\\&= \ket{\textcolor{blue}{1}0\textcolor{red}{1}}.\label{wrong2}
    \end{aligned}
\end{equation}

Now, there are two types of people in this world. The ones scandalized by what I just wrote and the ones thinking ``why is all this people so upset by it?''. To the people in the second group, I will explain. The numbering scheme we are using for the sites is arbitrary, this means, that physically the probability distribution should not change when we relabel the sites, in other words
\begin{equation}
    |\ket{\textcolor{red}{1}0\textcolor{blue}{1}}|^2 = |\ket{\textcolor{blue}{1}0\textcolor{red}{1}}|^2 ,
\end{equation}
this leads us to conclude that there can be at most a phase factor between both states
\begin{equation}
    \ket{\textcolor{red}{1}0\textcolor{blue}{1}} = e^{i\phi} \ket{\textcolor{blue}{1}0\textcolor{red}{1}},% \quad \phi = \{0, \pi\}.
\end{equation}
when the phase is $\phi=0$ we say our particles are bosons. On the other hand, when $\phi=\pi$ our particles are fermions. Now we see the problem, since our system is fermionic we should have
\begin{equation}
    \textcolor{blue}{c_3^\dagger}\textcolor{red}{c_1^\dagger}\ket{000} = - \textcolor{blue}{c_1^\dagger}\textcolor{red}{c_3^\dagger}\ket{000},
    \label{right1}
\end{equation}
in order to go from the state in eq. \ref{wrong1} to the one in \ref{wrong2} it is only necessary to swap particles, i.e. the blue one goes to the first site, and the red one to the third. And, as we discussed, this swapping should introduce a phase factor, a $-1$ in this case. This behaviour did not arise in eqs. \ref{wrong1}-\ref{wrong2}, as a matter of fact it looks like a boson system instead of a fermion one. So, where is the problem? Who is in charge of deciding this properties? Luckily for us, there are not a lot of objects here we can blame, the operators must be responsible, since they are the ones creating, destroying, and together ``moving" the particles. They must be wrong, but truth be told, they would be perfect for boson systems, we were just unlucky, since they are insufficient for our case. We are forced then, to look for suitable operators. As a hint we have that somewhere there must be a -1 involved. We can then modify our previous definitions in eqs. \ref{def1.1}-\ref{def1.2} introducing a phase factor as follows
\begin{equation}
  c_i^\dagger \ket{n_1,\ldots,n_i,\ldots,n_n} =(-1)^{\sum_{j<i}n_j} (1-n_i)\ket{n_1,\ldots,1_i,\ldots,n_n},
\end{equation}
where this term $\sum_{j<i}n_j$ is only counting the number of particles in states numbered lower than i. In order to grab this definition tighter, let's use it in our problematic couple of states
\begin{eqnarray}
    \textcolor{blue}{c_3^\dagger}\textcolor{red}{c_1^\dagger}\ket{000} = \textcolor{blue}{c_3^\dagger}\textcolor{red}{(-1)^{\sum_{j<1}n_j}}\textcolor{red}{(1-n_1)}\ket{\textcolor{red}{1}00} = \textcolor{blue}{c_3^\dagger}\ket{\textcolor{red}{1}00}= \textcolor{blue}{(-1)^{\sum_{j<3}n_j}(1-n_3)}\ket{\textcolor{red}{1}0\textcolor{blue}{1}}=-\ket{\textcolor{red}{1}0\textcolor{blue}{1}},\label{20}\\
    \textcolor{blue}{c_1^\dagger}\textcolor{red}{c_3^\dagger}\ket{000} = \textcolor{blue}{c_1^\dagger} \textcolor{red}{(-1)^{\sum_{j<3}n_j}(1-n_3)} \ket{00\textcolor{red}{1}} =  \textcolor{blue}{c_1^\dagger} \ket{00\textcolor{red}{1}} = \textcolor{blue}{(-1)^{\sum_{j<1}n_j}(1-n_1)} \ket{\textcolor{blue}{1}0\textcolor{red}{1}}= \ket{\textcolor{blue}{1}0\textcolor{red}{1}}.\label{21}\quad
\end{eqnarray}
Now we have a phase factor between both states, and our sleepless nights can end. The definition for the annihilation operator is similar
\begin{equation}
  c_i \ket{n_1,\ldots,n_i,\ldots,n_n} = (-1)^{\sum_{j<i}n_j}n_i\ket{n_1,\ldots,0_i,\ldots,n_n}.
\end{equation}\\

It might result beneficial for us to play around a little bit with these operators, for example, it could be interesting to see their commutation rules, i.e. what happens if we exchange the order of the operators in general. To do this, let's define the anti-commutator between two operators as 
\begin{equation}
  \{A,B\} = AB + BA.
\end{equation}
Guessing what is the anti-commutator $\{c_i^\dagger,c_j^\dagger\} $ is an easy task if we have \ref{20} and \ref{21}. By adding them we can guess it will be zero. Proving this in general is simple, let's assume without loss of generality that $j>i$

\begin{equation}
    \begin{aligned}[b]
        c_j^\dagger c_i^\dagger \ket{n_1,\ldots,n_i,\ldots,n_n} &= c_j^\dagger \left((-1)^{\sum_{k<i}n_k} (1-n_i)\ket{n_1,\ldots,1_i,\ldots,n_n} \right),\\
        &=(-1)^{\sum_{k<i}n_k} (1-n_i)c_j^\dagger\ket{n_1,\ldots,1_i,\ldots,n_n},\\
        &=(-1)^{\sum_{k<i}n_k} (1-n_i) (-1)^{\sum_{s<j}n_s} (1-n_j)\ket{n_1,\ldots,1_i,\ldots,1_j,\ldots,n_n},\\
        &=(-1)^{\sum_{k<i}n_k + \sum_{s<j}n_s} (1-n_i) (1-n_j)\ket{n_1,\ldots,1_i,\ldots,1_j,\ldots,n_n},\\
        &=(-1)^{\left(\sum_{k<i}n_k + \sum_{s\neq i<j}n_s\right) + 1} (1-n_i) (1-n_j)\ket{n_1,\ldots,1_i,\ldots,1_j,\ldots,n_n}.
    \end{aligned}
    \label{24}
\end{equation}
We must comment on the last step. Remember that the sum $\sum_{k<i}n_k $ represents the number of particles in sites with labels less than $i$. When the second operator ($c_j^\dagger$) is applied the number of particles has been increased because $c_i^\dagger$ created one in site i and $i<j$, we wanted to make this fact clear on the last step. Now let's do this the other way around

\begin{equation}
    \begin{aligned}[b]
        c_i^\dagger c_j^\dagger \ket{n_1,\ldots,n_i,\ldots,n_n} &= c_i^\dagger \left((-1)^{\sum_{s<j}n_s} (1-n_j)\ket{n_1,\ldots,1_j,\ldots,n_n} \right),\\
        &=(-1)^{\sum_{s<j}n_s} (1-n_j)c_i^\dagger\ket{n_1,\ldots,1_i,\ldots,n_n},\\
        &=(-1)^{\sum_{s<j}n_s} (1-n_j)(-1)^{\sum_{k<i}n_k} (1-n_i)\ket{n_1,\ldots,1_i,\ldots,1_j,\ldots,n_n},\\
        &=(-1)^{\sum_{s<j}n_s + \sum_{k<i}n_k} (1-n_j) (1-n_i)\ket{n_1,\ldots,1_i,\ldots,1_j,\ldots,n_n}
    \end{aligned}\label{25}
\end{equation}

Eqs. \ref{24} and \ref{25} are very similar. The only difference is the exponents of $-1$, let's see what happens for two cases, when $n_i = 0$ and when $n_i=1$. For the former we can agree that $\sum_{s\neq i<j}n_s = \sum_{s<j}n_s$ and as a consequence 
\begin{equation*}
    \begin{aligned}
        c_j^\dagger c_i^\dagger \ket{n_1,\ldots,n_i,\ldots,n_n} &=(-1)^{\left(\sum_{k<i}n_k + \sum_{s<j}n_s\right) + 1} (1-n_i) (1-n_j)\ket{n_1,\ldots,1_i,\ldots,1_j,\ldots,n_n},\\
        &=(-1)*(-1)^{\sum_{k<i}n_k + \sum_{s<j}n_s} (1-n_i) (1-n_j)\ket{n_1,\ldots,1_i,\ldots,1_j,\ldots,n_n},\\
        &=-(-1)^{\sum_{k<i}n_k + \sum_{s<j}n_s} (1-n_i) (1-n_j)\ket{n_1,\ldots,1_i,\ldots,1_j,\ldots,n_n},\\
        &= - c_i^\dagger c_j^\dagger \ket{n_1,\ldots,n_i,\ldots,n_n}.
    \end{aligned}
\end{equation*}
Hence, 
\begin{equation*}
    \begin{aligned}
        (c_j^\dagger c_i^\dagger + c_i^\dagger c_j^\dagger) \ket{n_1,\ldots,n_i,\ldots,n_n} &=0.
    \end{aligned}
\end{equation*}
If $n_i=1$, this las result is trivial, since the term $(1-n_i)=0$ making the right hand side  of eqs. \ref{24} and \ref{25} equal to zero.\\

So we obtain, finally, the relation 
\begin{equation}
  \{c_i^\dagger,c_j^\dagger\} = 0.
\end{equation}
A similar proof can be made for the anti-commutator
\begin{equation}
  \{c_i,c_j\} = 0.
\end{equation}

The only remaining relation to prove is, what happens when we mix the operators, namely what is $\{c_i^\dagger,c_j\}$. Let us prove this last relation for the general case again, just to exercise our pencils.\\

Assume that $i=j$, for this case
\begin{equation}
  \begin{aligned}[b]
    \{c_i^\dagger, c_i\}\ket{n_1,\ldots,n_n} &= (c_i^\dagger c_i + c_i c_i^\dagger)\ket{n_1,\ldots,n_n}, \\
    &= c_i^\dagger c_i \ket{n_1,\ldots,n_n} + c_i c_i^\dagger \ket{n_1,\ldots,n_n},\\
    &= c_i^\dagger (-1)^{\sum_{k<i}n_k}(n_i)\ket{n_1,\ldots,0_i,\ldots,n_n} + c_i (-1)^{\sum_{k<i}n_k}(1-n_i) \ket{n_1,\ldots,1_i,\ldots,n_n},\\
    &=  (-1)^{\sum_{k<i}n_k}(-1)^{\sum_{k<i}n_k}(1-0_i)(n_i)\ket{n_1,\ldots,1_i,\ldots,n_n} \\&+  (-1)^{\sum_{k<i}n_k}(-1)^{\sum_{k<i}n_k}(1_i)(1-n_i) \ket{n_1,\ldots,0_i,\ldots,n_n},\\
    &=  \left((-1)^{\sum_{k<i}n_k}\right)^2(n_i)\ket{n_1,\ldots,1_i,\ldots,n_n} \\&+  \left((-1)^{\sum_{k<i}n_k}\right)^2(1-n_i) \ket{n_1,\ldots,0_i,\ldots,n_n},\\
    &=  n_i\ket{n_1,\ldots,1_i,\ldots,n_n} + (1-n_i) \ket{n_1,\ldots,0_i,\ldots,n_n},
  \end{aligned}\label{ac}
\end{equation}
now if the original state had $n_i = 0$, from eq. \ref{ac} we get \begin{equation}
  \begin{aligned}[b]
    \{c_i^\dagger, c_i\}\ket{n_1,\ldots,0_i,\ldots,n_n}
    =  0\ket{n_1,\ldots,1_i,\ldots,n_n} + (1-0) \ket{n_1,\ldots,0_i,\ldots,n_n} =  \ket{n_1,\ldots,0_i,\ldots,n_n},
  \end{aligned}
\end{equation}
on the other hand, if $n_i = 1$
\begin{equation}
\begin{aligned}[b]
  \{c_i^\dagger, c_i\}\ket{n_1,\ldots,1_i,\ldots,n_n}
  =  1\ket{n_1,\ldots,1_i,\ldots,n_n} + (1-1) \ket{n_1,\ldots,0_i,\ldots,n_n} = \ket{n_1,\ldots,1_i,\ldots,n_n}
\end{aligned}
\end{equation}
therefore
\begin{equation}
  \{c_i^\dagger, c_i\} = 1.\label{31}
\end{equation}
Now for the case $i\neq j$ we have two possibilities, $i<j$ or $i>j$, we will prove it for the former, the proof for the case $i>j$ is analogous and writing it doesn't add new things to the discussion\footnote{Even though it can be a nice exercise.}
\begin{equation}
  \begin{aligned}[b]
    \{c_i^\dagger, c_j\}\ket{n_1,\ldots,n_n} &= (c_i^\dagger c_j + c_j c_i^\dagger)\ket{n_1,\ldots,n_n}, \\
    &= c_i^\dagger c_j \ket{n_1,\ldots,n_n} + c_j c_i^\dagger \ket{n_1,\ldots,n_n},\\
    &= c_i^\dagger (-1)^{\sum_{k<j}n_k}(n_j)\ket{n_1,\ldots,0_j,\ldots,n_n} \\
    &+ c_j (-1)^{\sum_{k<i}n_k}(1-n_i) \ket{n_1,\ldots,1_i,\ldots,n_n},\\
  \text{step 4 }  &= (-1)^{\sum_{k<i}n_k} (-1)^{\sum_{k<j}n_k}(1-n_i)(n_j)\ket{n_1,\ldots,1_i,\ldots,0_j,\ldots,n_n} \\
    &+ (-1)^{(\sum_{k<j}n_k) + 1} (-1)^{\sum_{k<i}n_k}n_j(1-n_i) \ket{n_1,\ldots,1_i,\ldots,0_j,\ldots,n_n},\\
    &= (-1)^{\sum_{k<i}n_k} (-1)^{\sum_{k<j}n_k}(1-n_i)(n_j)(1+(-1)^1)\ket{n_1,\ldots,1_i,\ldots,0_j,\ldots,n_n}\\
    &=0.
  \end{aligned}
  \label{32}
\end{equation}


In the second term of step 4, a $+1$ appears in the exponent due to the fact that one fermion was created in the previous step in position $i$, and since $i<j$ then $\sum_{k<j}n_k$ will be what it was, plus the fermion that got created. It is the same trick we used in eq. \ref{24} but explained in a different way. From eqs. \ref{31}, \ref{32} and the analogous eq. to \ref{32} when $i>j$ we can conclude that  $\{c_i^\dagger,c_j\} = \delta_{ij}$ as we wanted to prove.\\


\subsection{Dirac's equation}
This proof follows the one given by Griffiths \cite{griffiths}.\\
Let's start with the relativistic energy-momentum relation
\begin{equation}
  E^2 = p^2 c^2 + m^2 c^4,
\end{equation}
or in four-vector notation
\begin{equation}
  p^\mu p_\mu - m^2c^2 = 0.
  \label{covmom}
\end{equation}
And we, as Dirac, are looking for a way to factor this relation. Our lifes would be simple if we only had $p_0$, because in that case\footnote{Here we take the following convention. Greek indexes go from 0 to 3, and Roman go from 1 to 3.} $p_i = 0$, and as a consequence we can factor the equation trivially
\begin{equation*}
  (p_0)^2 - m^2 c^2 = (p^0 + mc)(p^0 - mc) = 0.
\end{equation*}
We went from a quadratic equation to two linear ones

\begin{eqnarray}
  p_0 + mc = 0,\\
  p_0 - mc = 0.
\end{eqnarray}
This is an easier problem. But in general we are not so lucky, in that case we need to tackle the problem when at least one component of the momentum is non-zero $p_i \neq 0$.\\

We are looking then for a way to factor eq. \ref{covmom} like this
\begin{equation*}
  p^\mu p_\mu - m^2c^2 = (1_\mu p^\mu+mc)(1^\nu p_\nu -mc),
\end{equation*}
where $1^\mu = (1,1,1,1)$. But there is a small problem with this equation, let's expand eq. \ref{covmom} to see it clearly
\begin{equation*}
  p^\mu p_\mu - m^2c^2 = p^0p_0 + p^1p_1 + p^2p_2 + p^3p_3 - m^2c^2,
\end{equation*}
on the other hand\footnote{Here we are using the metric $g_{\mu\nu},$ such that $g_{00} = 1, g_{ii} = -1,$ and $g_{\gamma \delta} = 0$ if $\gamma \neq \delta$.}

\begin{equation*}
  \begin{aligned}
    (p^\mu+mc)(p_\nu -mc) = &(p^0 - p^1 - p^2 - p^3 - mc)(p_0 - p_1 - p_2 - p_3 - mc), \\
    = &p^0p_0 + p^1p_1 + p^2p_2 + p^3p_3 - m^2c^2 \\
    &- p^0p_1 - p^0p_2 - p^0p_3 - p^1p_0 + p^1p_2 + p^1p_3 \\
    &- p^2p_0 + p^2p_1 + p^2p_3 - p^3p_0 + p^3p_1 + p^3p_2.
  \end{aligned}
\end{equation*}
As we can see we obtain crossed terms that are not supposed to be there. We could get rid of them by inserting some coefficients, like this

\begin{equation}
  p^\mu p_\mu - m^2c^2 = (\beta^\mu p_\mu + mc)(\gamma^\nu p_\nu -mc),
  \label{fact1}
\end{equation}
where we changed $p^\mu$ to $p_\mu$ for convenience, since that change can be absorbed in the $\beta^\mu$ coefficients. Let's see now what happens
\begin{equation*}
  (\beta^\mu p_\mu + mc)(\gamma^\nu p_\nu -mc) = \beta^\mu\gamma^\nu p_\mu p _\nu - mc(\beta^k - \gamma^k)p_k - m^2c^2,
\end{equation*}
we don't want the linear terms, this forces us to choose $\beta^\mu = \gamma^\mu$, we could be tempted to think that $\gamma^\mu = (1,1,1,1)$ solves the problem, but we need to remember that now both of the four-momentum vectors are covariant, originally one of them was covariant and the other contravariant, this introduces sign changes on the spatial coordinates\footnote{Just as a quick reminder $p^\mu = (p^0, p^1, p^2, p^3)$, $p_\mu = g_{\nu\mu}p^\nu = (p^0,-p^1,-p^2,-p^3)$.}. After all this, eq. \ref{fact1} becomes
\begin{equation}
  p^\mu p_\mu - m^2c^2 = (\gamma^\mu p_\mu + mc)(\gamma^\nu p_\nu -mc),
  \label{fact2}
\end{equation}
we look for $\gamma$ such that

\begin{equation*}
  (\gamma^\mu p_\mu + mc)(\gamma^\nu p_\nu -mc) = \gamma^\mu\gamma^\nu p_\mu p_\nu - m^2c^2 = p_\mu p_\nu - m^2c^2,
\end{equation*}
our problem will be solved then once we find $\gamma^\mu$ such that $\gamma^\mu\gamma^\nu p_\mu p_\nu = p_\mu p_\nu$, let's explore this equation further

\begin{equation*}
  \begin{aligned}
    \gamma^\mu\gamma^\nu p_\mu p_\nu = &(\gamma^\mu p_\mu)(\gamma^\nu p_\nu),\\
    =&(\gamma^0 p_0 + \gamma^1 p_1 + \gamma^2 p_2 + \gamma^3 p_3)(\gamma^0 p_0 + \gamma^1 p_1 + \gamma^2 p_2 + \gamma^3 p_3),\\
    = &(\gamma^0)^2 (p_0)^2 + (\gamma^1)^2 (p_1)^2 + (\gamma^2)^2 (p_2)^2 + (\gamma^3)^2 (p_3)^2 \\
    & (\gamma^0\gamma^1 + \gamma^1\gamma^0)p_0p_1
    +(\gamma^0\gamma^2 + \gamma^2\gamma^0)p_0p_2
    +(\gamma^0\gamma^3 + \gamma^3\gamma^0)p_0p_3\\
    &+(\gamma^1\gamma^2 + \gamma^2\gamma^1)p_1p_2
    +(\gamma^1\gamma^3 + \gamma^3\gamma^1)p_1p_3
    +(\gamma^2\gamma^3 + \gamma^3\gamma^2)p_2p_3.
  \end{aligned}
\end{equation*}

By looking at the quadratic terms we are inclined to guess $\gamma^0 = 1$, $\gamma^j = i$. But this guess falls apart when we realize the crossed terms don't vanish. We need a smarter guess. Dirac was in the same position as us at some point, but he was a very smart, and brave man. So he thought, \textit{what if instead of scalar coefficients I use matrices?} Why? because the crossed terms share a similar form $\gamma^\mu\gamma^\nu + \gamma^\nu\gamma^\mu$ and we can maybe take advantage of the fact that the matrix product does not commute in general. Our goal has changed then, now we want a set of matrices such that

\begin{equation*}
  (\gamma^0)^2= 1, \quad (\gamma^j)^2 = -1, \quad \gamma^\mu \gamma^\nu + \beta^\nu \beta^\mu = 0 \text{ for } \mu \neq \nu.
\end{equation*}
in short\footnote{Here the curly bracket denote the anticommutator $\{A,B\} = AB + BA$.}
\begin{equation}
  \{\gamma^\mu, \gamma^\nu\} = g^{\mu \nu}.
\end{equation}

Dirac found a set of matrices that solve the problem. The only detail is, they are $4x4$ matrices

\begin{equation}
  \gamma^0 = \begin{pmatrix}
    \sigma^0 & 0\\
    0 & \sigma^0
  \end{pmatrix} , \qquad
  \gamma^i = \begin{pmatrix}
    0 & \sigma^j\\
    -\sigma^j & 0
  \end{pmatrix}
\end{equation}
where $\sigma^\mu$ are 2x2 complex matrices, $\sigma^0 = \mathbb{1}$, $\sigma^j$ is a Pauli matrix, and $0$ is the 2x2 matrix filled with zeros.\\

Thus, we are ready to get Dirac equation, let's take again
\begin{equation*}
  p^\mu p_\mu - m^2c^2 = (\gamma^\mu p_\mu + mc)(\gamma^\nu p_\nu -mc) = 0,
\end{equation*}
and from the product take any factor and set it to zero
\begin{equation}
  \gamma^\mu p_\mu - mc = 0.
\end{equation}
We can make the substitution $p_\mu \rightarrow i\hbar \partial_\mu$ and apply it to $\psi$

\begin{equation}
  (\gamma^\mu (i\hbar \partial_\mu) - mc)\psi = i\hbar \gamma^\mu \partial_\mu \psi - mc\psi = 0.
\end{equation}

This is Dirac's equation in covariant form. A remark needs to me made at this point. $\psi$ is no longer a complex number, but a 4 element column matrix.

\begin{equation}
  \psi = \begin{pmatrix}
    \psi_1\\
    \psi_2\\
    \psi_3\\
    \psi_4
  \end{pmatrix}.
\end{equation}
In its non-covariant version, Dirac's equation reads
\begin{equation}
    i\hbar \frac{\partial}{\partial t} \psi = (c \bm{\alpha} \bm{p} + \beta mc^2)\psi
\end{equation}
It is important to remark that $\bm{\alpha}$ and $\beta$ are $4\times4$ matrices that don't necessarily commute, and they are not unique. Nevertheless, a form proposed by Dirac himself is 

\begin{equation}
    \alpha_i = \sigma_i \otimes \sigma_i  = \begin{pmatrix}
    0 & \sigma_i\\
    \sigma_i & 0
    \end{pmatrix}, \quad \beta = \sigma_z \otimes \mathbb{1}  = \begin{pmatrix}
    \mathbb{1} & 0\\
    0 & -\mathbb{1}
    \end{pmatrix}.
\end{equation}
Note that $\gamma^\mu = (\beta; \beta \bm{\alpha})$.\\

In order to make this easier we will adhere to the convention $\hbar = c = 1$, as a result, Dirac's equation in both forms will be

\begin{equation}
  (i \gamma^\mu \partial_\mu - m)\psi = 0\label{diracCov}
\end{equation}
\begin{equation}
      i\frac{\partial}{\partial t} \psi = ( \bm{\alpha} \bm{p} + \beta m)\psi\label{diracNonCov}
\end{equation}

\subsection{Antiparticles}
Most of this comes from Aguado's review \cite{aguado}.\\

As our next step we can look for eigenenergies of eq. \ref{diracNonCov}, in other words, solve

\begin{equation}
    \begin{pmatrix} 
      m \mathbb{1}_{2\times2} & \bm{\sigma}\bm{p}\\
      \bm{\sigma}\bm{p} & -m\mathbb{1}_{2\times2}
    \end{pmatrix}
    \begin{pmatrix}
     \phi_A\\
     \phi_B
    \end{pmatrix} = E\begin{pmatrix}
     \phi_A\\
     \phi_B
    \end{pmatrix},
    \label{diracMat}
\end{equation}
where

\begin{equation}
\begin{aligned}[b]
    \bm{\sigma}\bm{p} & = \sigma_x p_x + \sigma_y p_y + \sigma_z p_z,\\
    &= \begin{pmatrix} 0 & 1 \\ 1 & 0 \end{pmatrix} p_x + 
    \begin{pmatrix} 0 & -i \\ i & 0 \end{pmatrix} p_y + \begin{pmatrix} 1 & 0 \\ 0 & -1 \end{pmatrix} p_z\\
    &= \begin{pmatrix} 0 & p_x \\ p_x & 0 \end{pmatrix} + 
    \begin{pmatrix} 0 & -i  p_y \\ i p_y & 0 \end{pmatrix}  + \begin{pmatrix} p_z & 0 \\ 0 & -p_z \end{pmatrix}\\
    &= \begin{pmatrix} p_z & p_x -i  p_y\\ p_x + ip_y & -p_z \end{pmatrix} 
\end{aligned}
\end{equation}

So let's do it

\begin{equation*}
    \begin{vmatrix}
     (m - E) \mathbb{1}_{2\times2} & \bm{\sigma}\bm{p}\\
      \bm{\sigma}\bm{p} & (-m -E)\mathbb{1}_{2\times2}
    \end{vmatrix} = 0
\end{equation*}
expanding

\begin{equation*}
    \begin{vmatrix}
     m - E & 0 & p_z & p_x -i  p_y\\
     0 & m-E & p_x + ip_y & -p_z \\
     p_z & p_x -i  p_y & -m-E & 0 \\
     p_x + ip_y & -p_z & 0& -m-E\\
    \end{vmatrix} = 0
\end{equation*}
the equation we get out of it is\footnote{Thanks to Wolframalpha}
\begin{equation}
    (p_x^2 + p_y^2 + p_z^2 + m^2 - E^2)^2 = (\bm{p}^2 + m^2 - E^2)^2 = 0
\end{equation}
by inspection we can already check that the solutions are\footnote{It might be useful to note that we might already know these energies, if we kept using $c$ they would look like $E = \pm \sqrt{\bm{p}^2c^2 + m^2c^4}$, familiar equations, specially for the case of a particle at rest (i.e. $\bm{p} = 0$) $E = \pm \sqrt{m^2c^4} = \pm mc^2$}
\begin{equation*}
\begin{aligned}
  E_1 &= \sqrt{\bm{p}^2 + m^2}\\
  E_2 &= \sqrt{\bm{p}^2 + m^2}\\
  E_3 &= -\sqrt{\bm{p}^2 + m^2}\\
  E_4 &= -\sqrt{\bm{p}^2 + m^2}
  \end{aligned}
\end{equation*}
for the moment we are not interested in the general eigenstates, but it might be worthwhile to analyse the case for particles at rest (i.e. $\bm{p}=0$), for this case the matrix in eq. \ref{diracMat} is already diagonal, therefore the eigenvalues are the diagonal elements, hence we obtain two particles with energy $E_+ = m$ and two with energy, $E_- = -m$, as for the eigenvectors they are

\begin{equation*}
    \psi_1 = \begin{pmatrix} 1 \\0 \\0 \\0\end{pmatrix}, \quad 
    \psi_2 = \begin{pmatrix} 0 \\1 \\0 \\0\end{pmatrix}, \quad 
    \psi_3 = \begin{pmatrix} 0 \\0 \\1 \\0\end{pmatrix}, \quad 
    \psi_4 = \begin{pmatrix} 0 \\0 \\0 \\1\end{pmatrix}.
\end{equation*}
In principle, having negative energy solutions does not make sense, but instead of disregarding this parts Dirac proposed they are also physical solutions corresponding to half spin particles with negative energies. One of the things this result tells us is that, an electron produced from the vacuum should be accompanied by a hole with negative energy.\\

\colorbox{red}{13.08.19 Insert the discussion about holes here}\\

\colorbox{red}{13.08.19 I still have to talk about spinors before, to explain why the solutions are spin up and spin down}

\subsection{Majorana equation}

Let's go back to eq. \ref{diracCov}, as we can notice, the elements of $\gamma^\mu$ can be complex. But what happens if we constrain Dirac's equation to real value? In other words, can we find $\gamma^\mu$ such that these matrices are purely imaginary?\footnote{Remember they appear in the term $i\gamma^\mu\partial_\mu$ of Dirac's equation.} Ettore Majorana in 1937 \cite{Majorana37} discovered a set of purely imaginary gamma matrices $\tilde{\gamma}^0 = \sigma_y \otimes \sigma_x$, $\tilde{\gamma}^1 = i\sigma_x\otimes \mathbb{1}$, $\tilde{\gamma}^2 = i\sigma_z\otimes \mathbb{1}$, $\tilde{\gamma}^3 = i\sigma_y\otimes \sigma_y$, this leads to
\begin{equation}
    (i\tilde{\gamma}^\mu\partial_\mu - m)\tilde{\psi} = 0,\label{majDirac}
\end{equation}
since the matrices $i\tilde{\gamma}^\mu$ are purely real this forces $\tilde{\psi}$ to be also purely real. This leads to the condition
\begin{eqnarray}
  \tilde{\psi} = \tilde{\psi}^*.
\end{eqnarray}
This is the so-called reality condition. Since an electrically charged particle is different from its antiparticle this means that the any particle resulting from eq. \ref{majDirac} has to be electrically neutral and equal to its own antiparticle.\\

\colorbox{red}{13.08.19 The ideas are there, but very rough. It needs to be rewritten}

\subsection{Charge conjugation}

If we want to approach formally the result of a particle being its own antiparticle we have to impose charge conjugation symmetry, this means that we will assume that the charge conjugated particle is equal to the original one and see what consequences does it have. In order to make the charge explicit in Dirac's equation we have to couple it to an electromagnetic field $A_\mu$ by making the substitution $i\partial_\mu \rightarrow i\partial_\mu + eA_\mu$. As a consequence, Dirac's equation is modified to
\begin{equation}
    [\gamma^\mu(i\partial_\mu + eA_\mu) - m]\psi = 0.
    \label{dir1}
\end{equation}
The charge conjugated solution $\psi_c$ should satisfy the same equation but with the sign of the electric charge flipped, hence
\begin{equation}
    [\gamma^\mu(i\partial_\mu - eA_\mu) - m]\psi_c = 0.
    \label{dir2}
\end{equation}
We want to see what is the connection between $\psi$ and $\psi_c$ and how can we arrive to the result $\psi = \psi_c$. So let's start by taking the complex conjugate of eq. \ref{dir1}
\begin{equation}
    [-\gamma^{\mu*}(i\partial_\mu - eA_\mu) - m]\psi^* = 0.
    \label{dir1.1}
\end{equation}
Now we define a matrix $\mathcal{C}$ such that $-(\mathcal{C}\gamma^0)\gamma^{\mu*} = \gamma^\mu(\mathcal{C}\gamma^0)$, then we can rewrite eq. \ref{dir1.1} to look like eq. \ref{dir2}
\begin{equation}
    [\gamma^\mu(i\partial_\mu - eA_\mu) - m]\mathcal{C}\gamma^0\psi^* = 0.
\end{equation}
So according to this, we can define charge conjugation as\footnote{Check this link, it has a nice explanation \url{https://physics.stackexchange.com/questions/48334/charge-conjugation-in-dirac-equation}} $\psi = \mathcal{C}\gamma^0 \psi^*$. The operator $\mathcal{C}$ is not unique, a nice choice for our case is $\mathcal{C} = i\gamma^2$ where
\begin{equation}
    i\gamma^2 = \begin{pmatrix}
    0 & 0 & 0 & -i \\
    0 & 0 & i & 0 \\
    0 & -i & 0 & 0 \\
    i & 0 & 0 & 0
    \end{pmatrix}.
\end{equation}
In our new language, then, when we say a particle is equal to its antiparticle we mean

\begin{eqnarray}
  \psi = \psi_c = i\gamma^2\psi^*.
\end{eqnarray}
This is sometimes called the pseudo-reality condition \cite{Jackiw2012}. There is another representation (i.e. election of the $\gamma^\mu$ matrices) that can help us to get the final insight we want about the mathematical structure necessary to have this Majorana particles (from now on we will call Majorana particles to all particles such that they are their own antiparticle). The choice this time is

\begin{equation}
    \alpha_i = \begin{pmatrix}
    -\sigma_i & 0\\
    0 & \sigma_i
    \end{pmatrix}, \beta = \begin{pmatrix}
     0 & \mathbb{1}\\
    \mathbb{1} & 0
    \end{pmatrix},
\end{equation}
which leads to
\begin{equation}
    \gamma^0 = \begin{pmatrix}
    0 & \mathbb{1}\\
    \mathbb{1} & 0
    \end{pmatrix}, \gamma^i = \begin{pmatrix}
    0 & \sigma_i\\
    -\sigma_i & 0
    \end{pmatrix}.
\end{equation}
This is known as the Weyl representation, and for this representation Dirac's equation is 
\begin{equation}
    \begin{pmatrix}
     -m & i\partial_t -\bm{\sigma p} \\ i\partial_t +\bm{\sigma p} & -m
    \end{pmatrix}\begin{pmatrix} \psi_L\\ \psi_R \end{pmatrix} = 0,
\end{equation}
for this case, the pseudo-reality condition is rewritten as 
\begin{equation}
    \begin{pmatrix} \psi_L\\ \psi_R \end{pmatrix} = \begin{pmatrix}
    0 & i\sigma_y\\
    -i\sigma_y & 0
    \end{pmatrix}\begin{pmatrix} \psi_L^*\\ \psi_R^* \end{pmatrix} = 
    \begin{pmatrix} i\sigma_y\psi_R^*\\ -i\sigma_y\psi_L^* \end{pmatrix}. 
\end{equation}
This decouples then to two equations
\begin{equation}
  (i\partial_t + \bm{\sigma p})\psi_L + im\sigma_y\psi_L^* = 0, \quad
  (i\partial_t - \bm{\sigma p})\psi_R - im\sigma_y\psi_R^* = 0.\label{weylpseudo}
\end{equation}
The fact that in both these equations we have coupling of particles and antiparticles has as a consequence that the conservation of charge is not given. Formally this means that the gauge invariance $\psi(r) \rightarrow e^{i\theta}\psi(r)$ is not present, this implies that Majorana particles don't couple to the electromagnetic field, and are necessarily charge neutral.\textcolor{red}{13.08.19 Look for a better explanation of this.}\\

% Finally we will get a profound result by analysing the stationary solutions to eq. \ref{weylpseudo}. Let's remember that this solutions are of the form $\phi_E = e^{iEt}\psi$. For each one of these solutions there exist a charge conjugated counterpart at negative energy $\phi_{-E}$, this is
% \begin{equation}
%     \phi_{-E} = \mathcal{C}\psi_E^*
% \end{equation}
% now we can build a field consisting of these modes
% \begin{equation}
%     \hat{\Psi}(\bm{r},t) = \sum_{E>0} a_E e^{-iEt}\phi_E + \sum_{E<0} b_{-E}^\dagger e^{-iEt}\phi_{-E},
% \end{equation}
% where $a_E$ is the annihilation operator for particles at energy $E$, and $b_{-E}^\dagger$ is the creation operator for antiparticles at energy $-E$. We can rewrite this equation as
% \begin{equation}
%     \hat{\Psi}(\bm{r},t) = \sum_{E>0} a_E e^{-iEt}\phi_E +  b_{E}^\dagger e^{iEt}\mathcal{C}\phi_{E}^*,
% \end{equation}
% by demanding $a_E = b_E$, we get the final field operator for a Majorana particle
% \begin{equation}
%     \hat{\Psi}(\bm{r},t) = \sum_{E>0} a_E e^{-iEt}\phi_E +  a_{E}^\dagger e^{iEt}\mathcal{C}\phi_{E}^*,
% \end{equation}
% \begin{equation*}
%     \mathcal{C} = \begin{pmatrix}
%     0 & -i\bm{\sigma}^2 \\ 
%     i\bm{\sigma}^2 & 0
%     \end{pmatrix}
% \end{equation*}

\newpage
\section{Colloqium: Majorana fermions in nuclear, particle, and solid-state physics}
\input{parts/revModPhys.tex}

\newpage
\section{Unpaired Majorana fermions in quantum wires}
Original paper \cite{2001kitaev}.\\

The idea of this paper is to construct a system such that we can have two isolated states immune to decoherence. This type of systems can be extremely useful for quantum computing. If qbits can not lose coherence, then there is no necessity of implementing correction algorithms, and this would lead to simpler and faster implementations of quantum algorithms.\\

According to Kitaev, there is two kind of errors that can affect a quantum state

\begin{enumerate}
    \item Classical errors that flip the value of a qbit from $\ket{0}$ to $\ket{1}$.
    \item Phase errors that change the sign of all the states with the jth qbit equal to $\ket{1}$ with respect to the ones that have a jth qbit equal to $\ket{0}$.
\end{enumerate}

It seems to be the case that usually getting rid of one of those error sources is simple, but the problem complicates when both need to be suppressed.\\

Kitaev then proceeds to mentally set a system that would be a candidate for eliminating these errors. Let's imagine an isolated\footnote{Do we need the system to be isolated? I think it should be sufficient if we ask for the system to conserve fermionic charge.} one dimensional system consisting of fermionic sites that can be either empty ($\ket{0}$) or occupied ($\ket{1}$). Classical errors are impossible to happen only in one site, this would imply that either one fermion got annihilated, or got created out of nowhere. But there are mechanisms that would make possible for two classical errors to happen simultaneously, let's say for example that we have a state $\ket{01}$\footnote{This means that the first state is $\ket{0}$ and the second $\ket{1}$.} it can happen that the fermion of the site on the right hopps to the left and changes the global state of the system to $\ket{10}$. As we can see, for this case we have two classical errors simultaneously, while preserving fermionic charge. The reader might get creative here and think, if this error arises from an electron hopping from one site to the other, then we can get rid of this error if we make really hard for the fermion to hop. This could happen for example in a system in which both sites are far apart, let's say we have now $\ket{0\ldots1}$ and the intermediate states are somehow fixed, so the only possible interaction between the state at the beginning and the one at the end is direct. We can try to find a system such that the hopping probability decreases rapidly as we move away from the site. This is what we are going to do now.\\

\colorbox{yellow}{28.11.18 At some point this section need to be finished. This can be done once the material is enough to }\\
\colorbox{yellow}{decipher Kitaev's paper.}


\newpage
\section{Majorana Qubits}
Corresponds to this reference \cite{2014Hassler}. To be honest, I use it as a guideline, but all this can be considered as my own writting.\\

\subsection{Introduction}

Let's say we have a piece of cloth, very elastic, and additionally it is impossible to make holes on it or rip it apart. We can stretch it in one direction, in the other, maybe in both. Twist it, bend it, any contortion or elongation; the only forbidden deformations are any kind of ripping on the cloth. There are ways to analyze all the physical phenomena playing a role while we deform the cloth, but at the end of the day we can with all certainty keep saying that what we have in our hands is (and throughout the process was) still an elastic piece of cloth. This leads us to think there are some properties that were not changed in the process: the color, the texture, the pattern, and so on. That is to say, we know that some properties were unchanged under all the deformations, we can use our senses to perform an analysis. For mathematical objects, since they are abstract, we don't have the advantage of directly using our senses, therefore we need a tool, topology is the one in charge of analyzing the properties that remain unchanged as we stretch or bend such bodies.\\

Topology became a major branch of mathematics during the 20th century, but some of its ideas go back as early as 1736 with the seven bridges of Könisberg problem \cite{konigs}. But for a long time, there was not an obvious way to make use of it in physics or chemistry, and as a consequence it was for a long time used exclusively by mathematicians. In the last decade it has been found that topology is a very useful tool to gain insight into the physics of materials \cite{natTop}. In this section it will be shown how can we use topology to build a system whose applications can potentially lead to an implementation of quantum computing that is very resistant to certain kind of errors.

\subsection{Second quantization and Majorana fermions}

So far we have being treating the operator $c_j$ as an object which destroys a particle located in site $j$, but we can also think of it as the creator of an antiparticle. For example it can create an positron, antiproton, antineutron, etc. When such antiparticles come into contact with their counterparts, namely electron, proton, neutron ..., both annihilate each other. Each particle has it's own antiparticle, this is a direct result of Dirac's equation. Each superhero has it's own villain, but can a superhero be also the villain? Next section will dive further into this idea.

\colorbox{orange}{30.04.19 Rethink the last paragraph, the idea is there, but it might not be well expressed.}

\subsubsection{Majorana fermions}

It was not long since Schrödinger developed his celebrated wave equation that people realized it was not relativistic, i.e. it was not Lorentz invariant. This set the next step for the theoretical community, find a relativistic Schrödinger equation. In 1926 Oskar Klein and Walter Gordon propose a Lorentz invariant version of it, however, it has two important flaws, it could not be used for spin half particles, and as a byproduct had a solution with negative energy\cite{antimatter}. Dirac went on looking for a equation that could be used with fermions (spin half particles), eventually he arrived to the equation discussed in section \ref{dirEq}. This one, nevertheless, kept one of the Klein-Gordon equation flaws, namely, the negative energy solution. Dirac was less shy on this affair, instead of labeling this solution as unphysical, he said this showed that each particle has a mirror antiparticle with almost similar properties, except for the electrical charge. This would not be the only surprise we were going to get from Dirac's equation as we will see soon.\\


Ettore Majorana found another big surprise hidden within Dirac's equation while forcing it to have purely real solutions. For this case he found that the resulting wave functions characterized particles that were at the same time their own antiparticles. As a tribute to him, they were baptized as Majorana fermions or Majorana particles. So in a sense, creating a Majorana particle would be equivalent to creating its antiparticle. Mathematically, we can represent this property through operators as $\gamma_i = \gamma_i^\dagger$. As their fermionic counterparts they also have anticommutation relations, in this case they are 
\begin{equation}
    \{\gamma_k,\gamma_j\} = 2\delta_{kj},
    \label{anticomm}
\end{equation}
where the factor of 2 is there just for convenience as we will see later.\\

Since Majorana's prediction in 1937 there has been and intense search for these fermions in the wild. But so far, nothing conclusive has been obtained. This then lead us to a trivial and a not so trivial question. The former would be: if they are so elusive, why do we care about them? The fact that they have proved to be hard to find might dissuade us from trying to find applications for them. The second question is slightly related to the former, if this fermions cannot be found as elementary particles, is there a way to create them? We can for example look for them as excitations in materials, as with phonons, magnons, plasmons, etc. This approach has advantages, for example it can provide us with ways to create, destroy, and control these particles. From this point on, we will use the term Majorana modes when we refer to Majorana fermions arising from excitations.\\

At this point the reader might suspect that the author has a secret fascination for Majorana modes and wants to see them at any cost. Truth be told, suspicions are almost correct (the fascination is not secret though), however, it is out task to clarify why they can be of importance, and with a little bit of luck encourage the reader to travel this path too. The remaining part of this subsection will be dedicated to lay down the mathematical foundations needed to understand and manipulate these objects.\\

For the fermionic creation and annihilation operators we were able to go from their definition to the anticommutation relations they comply with. But know we find ourselves in a different position, namely we know the commutation relations and a simple property about the operators. We can give this a thought, could we construct the Majorana fermions creation and annihilation operators solely with this properties? We don't have to start from zero, we already know some things, for example the fermionic operators $c_i$ and $c_i^\dagger$, maybe they can help us. Let's write an arbitrary linear combination of both

\begin{equation*}
    \begin{aligned}
        &\gamma_k &= |a|e^{i\phi}c_k + |b|e^{i\omega}c_k^\dagger,
    \end{aligned}
\end{equation*}
given this, the hermitian conjugate operator would be 
\begin{equation*}
    \begin{aligned}
        \Rightarrow\quad & \gamma_k^\dagger &= (|a|e^{i\phi}c_k + |b|e^{i\omega}c_k^\dagger)^\dagger,\\
        && = |a|e^{-i\phi}c_k^\dagger + |b|e^{-i\omega}c_k.
    \end{aligned}
\end{equation*}
\begin{wrapfigure}{l}{0.5\textwidth} % Inline image example
   \begin{center}
     \resizebox {0.35\columnwidth} {!} {
    \begin{tikzpicture}
    \begin{scope}[thick,font=\scriptsize]
    %xscale=2
    % Axes:
    % Are simply drawn using line with the `->` option to make them arrows:
    % The main labels of the axes can be places using `node`s:
    \draw [->] (-1.9,0) -- (1.9,0) node [right]  {$\Re\{z\}$};
    \draw [->] (0,-1.9) -- (0,1.9) node [above] {$\Im\{z\}$};

    % Axes labels:
    % Are drawn using small lines and labeled with `node`s. The placement can be set using options
    \iffalse% Single
    % If you only want a single label per axis side:
    \draw (1,-3pt) -- (1,3pt)   node [above] {$1$};
    \draw (-1,-3pt) -- (-1,3pt) node [above] {$-1$};
    \draw (-3pt,1) -- (3pt,1)   node [right] {$i$};
    \draw (-3pt,-1) -- (3pt,-1) node [right] {$-i$};
    \else% Multiple
    % If you want labels at every unit step:
    \draw (1,-4pt) -- (1,4pt)   node [right] {\textcolor{red}{$\phi = 0$}};
    \draw (-4pt,1) -- (4pt,1)   node [right] {\textcolor{red}{$\phi = \pi/2$}};
    \draw (-4pt,-1) -- (4pt,-1)   node [right] {\textcolor{red}{$\phi = 3\pi/2$}};
    \draw (-1,-4pt) -- (-1,4pt)   node [left] {\textcolor{red}{$\phi = \pi$}};

    \draw (1,-4pt) -- (1,4pt)   node [left] {$1$};
    \draw (-4pt,1) -- (4pt,1)   node [below] {$i$};
    \draw (-4pt,-1) -- (4pt,-1)   node [above] {$-i$};
    \draw (-1,-4pt) -- (-1,4pt)   node [right] {$-1$};

    \fi
    \end{scope}
    % The circle is drawn with `(x,y) circle (radius)`
    % You can draw the outer border and fill the inner area differently.
    % Here I use gray, semitransparent filling to not cover the axes below the circle
    \path [draw=none,fill=gray,semitransparent] (0,0) circle (1);
\end{tikzpicture}
}
   \end{center}
   \caption{Unitary circle in the complex plane. In red, the corresponding angles of the polar representation $e^{i\phi}$.}
   \label{unitCircle}
 \end{wrapfigure}
We mentioned earlier that the condition of a Majorana fermion being its own antiparticle would translate mathematically to the equation $\gamma_i = \gamma_i^\dagger$. This leads us to 
\begin{equation*}
    \begin{aligned}
     |a|e^{i\phi}c_k + |b|e^{i\omega}c_k^\dagger &= |a|e^{-i\phi}c_k^\dagger + |b|e^{-i\omega}c_k,\\
     \Rightarrow \quad|a|e^{i\phi} &= |b|e^{-i\omega},\\
     \Rightarrow \quad|b|e^{i\omega} &= |a|e^{-i\phi}.
    \end{aligned}
\end{equation*}
From last two equalities we can conclude two things: $|a| = |b|$ and $\phi = -\omega$. As a consequence our assumptions for the $\gamma_k$ operators get modified to 
$$\gamma_k = |a|e^{i\phi}c_k + |a|e^{-i\phi}c_k^\dagger = |a|(e^{i\phi}c_k + e^{-i\phi}c_k^\dagger).$$
Hence the property $\gamma_i = \gamma_i^\dagger$ is properly covered. Now is time to take care of the anticommutation relations in eq. \ref{anticomm}, so let's calculate\\

 \begin{equation}
    \begin{aligned}[b]
     \{\gamma_k,\gamma_l\} &= \{ |a|(e^{i\phi_k}c_k +  e^{-i\phi_k}c_k^\dagger), |a|(e^{i\phi_l}c_l +  e^{-i\phi_l}c_l^\dagger)\}, \\
     &= |a|^2(\{ e^{i\phi_k}c_k, e^{i\phi_l}c_l\} +  \{ e^{i\phi_k}c_k , e^{-i\phi_l}c_l^\dagger\} +  \{e^{-i\phi_k}c_k^\dagger, e^{i\phi_l}c_l\} +  \{e^{-i\phi_k}c_k^\dagger, e^{-i\phi_l}c_l^\dagger\}),\\
     &= |a|^2(e^{i(\phi_k+\phi_l)}\cancelto{0}{\{ c_k,c_l\}} +  e^{i(\phi-\phi_l)}\{ c_k ,c_l^\dagger\} +  e^{i(-\phi_k+\phi_l)}\{c_k^\dagger, c_l\} +  e^{i(-\phi_k-\phi_l)}\cancelto{0}{\{c_k^\dagger,c_l^\dagger\}}),\\
     &= |a|^2(e^{i(\phi_k-\phi_l)}\cancelto{\delta_{kl}}{\{ c_k ,c_l^\dagger\}} +  e^{i(-\phi_k+\phi_l)}\cancelto{\delta_{kl}}{\{c_k^\dagger, c_l\}}),\\
     &= |a|^2(e^{i(\phi_k-\phi_l)}\delta_{kl} +  e^{i(-\phi_k+\phi_l)}\delta_{kl}).
    \end{aligned}
    \label{34}
\end{equation}
Given last equation, if we set $k=l$ we get that $\{\gamma_k,\gamma_k\} = |a|(e^{i(\phi_k-\phi_k)}\delta_{kk} +  e^{i(-\phi_k+\phi_k)}\delta_{kk}) = 2|a| $. For $k \neq l$ we have $\{\gamma_k,\gamma_l\} = 0$, since $\delta_{kl} = 0$ for this case. This means that we can write our result in a simpler way, that is to say eq. \ref{34} is equivalent to 
\begin{equation*}
    \{\gamma_k,\gamma_l\} = e^{i(\phi_k-\phi_l)}\delta_{kl} +  e^{i(-\phi_k+\phi_l)}\delta_{kl} = 2|a| \delta_{kl}
\end{equation*}
Since these operators have to satisfy the anticommutation relations in eq. \ref{anticomm} this imposes $|a|=1$, then $\gamma_k$ mutates to 
 \begin{equation*}
    \gamma_k = e^{i\phi}c_k + e^{-i\phi}c_k^\dagger.
\end{equation*}
To give a quick rundown. Any operator of this form meets the requirements to be a Majorana operator, i.e. the anticommutator and being its own hermitian conjugate. If we look carefully at this last equation we might realize that we face an overchoice. We are left with the freedom to select any value for $\phi$. This is not a problem per se, nevertheless we have to consider if some options are better than others, and for this it can be beneficial to explore a couple of special cases. If we look at fig. \ref{unitCircle} we can see that there are some special instances for this phase factor, $\phi = 0,\pi$, and $\phi = \pi/2, 3\pi/2$. $\phi = 0, \pi$ results in
\begin{equation*}
    \gamma_k = e^{i0}c_k + e^{-i0}c_k^\dagger = c_k + c_k^\dagger, \qquad \gamma_k = e^{i\pi}c_k + e^{-i\pi}c_k^\dagger = -(c_k + c_k^\dagger),
\end{equation*}
these are the only two combinations that give purely real operators. In fact, it can be considered as just one instance, since the only difference between the two of them is a global phase (in this case $e^{i\pi}$). Furthermore, we have the option $\phi = \pi/2, 3\pi/2$, this one leads to
\begin{equation*}
    \gamma_k = e^{i\pi/2}c_k + e^{-i\pi/2}c_k^\dagger = ic_k -i c_k^\dagger = i(c_k - c_k^\dagger), \qquad \gamma_k = e^{i3\pi/2}c_k + e^{-i3\pi/2}c_k^\dagger = -i(c_k -c_k^\dagger),
\end{equation*}
this time, we get purely imaginary operators. And as in the previous case the only difference between this pair is a global $e^{i\pi}$ phase.\\

There might be extra mysteries hidden within this objects. Once again let's try to start from known properties of $c_k$, $c^\dagger_k$ and see what happens when extrapolated to $\gamma_k$. One of the most important composed operators is, without a doubt the number operator $c_k^\dagger c_k$. Its eigenvalue is the number of particles existent in site $k$, let's see this mathematicallly
\begin{equation*}
    \begin{aligned}
    c_k^\dagger c_k \ket{n_1,\ldots,n_k,\ldots,n_n} &= c_k^\dagger (-1)^{\sum_{j<k}n_j} n_k \ket{n_1,\ldots,0_k,\ldots,n_n},\\
    &=  (-1)^{\sum_{j<k}n_j} n_k (c_k^\dagger \ket{n_1,\ldots,0_k,\ldots,n_n}),\\
    &=  (-1)^{\sum_{j<k}n_j} n_k ((-1)^{\sum_{j<k}n_j} (1-0) \ket{n_1,\ldots,1_k,\ldots,n_n}),\\
    &=  ((-1)^{\sum_{j<k}n_j})^2 n_k \ket{n_1,\ldots,1_k,\ldots,n_n},\\
    &= n_k \ket{n_1,\ldots,1_k,\ldots,n_n}.
    \end{aligned}
\end{equation*}
The only thing that might bother us is that in the last equality we have the state $\ket{n_1,\ldots,1_k,\ldots,n_n}$ with $1$ in the k-th position, this can lead us to think that it only works when $n_k=1$, but this is true also for $n_k=0$, as we prove below
\begin{equation*}
    \begin{aligned}
    c_k^\dagger c_k \ket{n_1,\ldots,0_k,\ldots,n_n}  &= 0 \ket{n_1,\ldots,1_k,\ldots,n_n},\\
    &= 0 ,\\
    &= 0 \ket{n_1,\ldots,0_k,\ldots,n_n}.
    \end{aligned}
\end{equation*}
We can agree now, that $c_k^\dagger c_k = n_k$. There can be also different combinations of these operators, for example $c_kc_k$, and $c_k^\dagger c_k^\dagger$. This is a nice opportunity to test our knowledge about some of the anticommutation relations we already know, so let's do it just for fun
\begin{eqnarray}
    \{c_k,c_j\} = 0 \Rightarrow \{c_k,c_k\} = c_kc_k + c_kc_k = 2c_k = 0 \Rightarrow c_kc_k = 0,\nonumber\\
    \{c_k^\dagger, c_j^\dagger\} = 0 \Rightarrow  \{c_k^\dagger,c_k^\dagger\} = c_k^\dagger c_k^\dagger + c_k^\dagger c_k^\dagger = 2c_k^\dagger c_k^\dagger =0 \Rightarrow c_k^\dagger c_k^\dagger =0. \nonumber
\end{eqnarray}
This actually makes sense, we cannot destroy a particle twice ($c_kc_k$) so this result must lead to zero, and we cannot have two fermions in the same site ($c_k^\dagger c_k^\dagger$), they hate being together, therefore this must lead to zero as well. Now, can we write the number operator with $\gamma_k$? If we were to guess from what we know we might be tempted to propose $\gamma_k^\dagger \gamma_k$. Let's see what happens with this definition.
\begin{equation}
\begin{aligned}[b]
    \gamma_k^\dagger \gamma_k = \gamma_k\gamma_k = \gamma_k^2 &= (e^{i\phi}c_k + e^{-i\phi}c_k^\dagger)^2, \\
    &= e^{i2\phi}\cancelto{0}{c_kc_k} + c_k c_k^\dagger + c_k^\dagger c_k + e^{-i2\phi}\cancelto{0}{c_k^\dagger c_k^\dagger},\\
    &= c_k c_k^\dagger + c_k^\dagger c_k,\\
    &= \{c_k, c_k^\dagger\},\\
    &= \{c_k^\dagger, c_k\}, \\
    & = \delta_{kk},\\
    &= 1.
    \end{aligned}
    \label{quadOne}
\end{equation}
This is clearly not the number operator. So we have to keep looking for it, should we try random combinations until one of them works? That depends on how much time do we have, for example, the author needs to finish this thesis, so in this case the answer is a big ``no". We can take a smarter approach, we know one form of the number operator ($c_k^\dagger c_k$), and we know that we can write $\gamma_k$ as linear combination of $c_k$, and $c_k^\dagger$. But if we could go back, this is, what if we could write the $c_k$ as combination of $\gamma_k$ operators. Given this we would have no problem to give an expression for the number operator in terms  of $\gamma_k$. Let's then try to recover $c_k$ from two arbitrary majorana operators
\begin{equation*}
    \begin{aligned}
        c_k &= |a|e^{i\theta} \gamma_k + |b|e^{i\chi}\gamma_k',\\
        &=|a|e^{i\theta} (e^{i\phi}c_k + e^{-i\phi}c_k^\dagger) + |b|e^{i\chi} (e^{i\phi'}c_k + e^{-i\phi'}c_k^\dagger),\\
        &= c_k\left(|a|e^{i\theta} e^{i\phi} + |b|e^{i\chi}e^{i\phi'}\right) + c_k^\dagger \left(|a|e^{i\theta}e^{-i\phi} +  |b|e^{i\chi}e^{-i\phi'}\right),\\
        &= c_k\left(|a|e^{i(\theta+\phi)} + |b|e^{i(\chi+\phi')}\right) + c_k^\dagger \left(|a|e^{i(\theta-\phi)} +  |b|e^{i(\chi-\phi')}\right).\\
    \end{aligned}
\end{equation*}
\begin{wrapfigure}[14]{l}{0.40\textwidth} % Inline image example
   \begin{center}
     \resizebox {0.35\columnwidth} {!} {
    \begin{tikzpicture}
    \begin{scope}[thick,font=\scriptsize]
    %xscale=2
    % Axes:
    % Are simply drawn using line with the `->` option to make them arrows:
    % The main labels of the axes can be places using `node`s:
    \draw [->] (-1.9,0) -- (1.9,0) node [right]  {$\Re\{z\}$};
    \draw [->] (0,-1.9) -- (0,1.9) node [above] {$\Im\{z\}$};
    % Axes labels:
    % Are drawn using small lines and labeled with `node`s. The placement can be set using options

    % If you only want a single label per axis side:
    \draw [->] (0,0) -- (0.70710678118654746,0.70710678118654746) node [right]  {\textcolor{red}{$e^{i\phi}$}};
    \draw [->] (0,0) -- (-0.70710678118654746,-0.70710678118654746) node [left]  {\textcolor{red}{$e^{i(\phi+\pi)}$}};
    \draw [->] (0,0) -- (0.70710678118654746,-0.70710678118654746) node [right]  {\textcolor{red}{$e^{-i\phi}$}};

    \end{scope}
    % The circle is drawn with `(x,y) circle (radius)`
    % You can draw the outer border and fill the inner area differently.
    % Here I use gray, semitransparent filling to not cover the axes below the circle
    \path [draw=none,fill=gray,semitransparent] (0,0) circle (1);
\end{tikzpicture}
}
   \end{center}
   \caption{Unitary circle in the complex plane. In red, the corresponding angles of the polar representation $e^{i\phi}$.}
   \label{circle2}
 \end{wrapfigure}
From this equation we can quickly obtain some restrictions, namely
\begin{eqnarray}
    |a|e^{i(\theta+\phi)} + |b|e^{i(\chi+\phi')} = 1,\label{first}\\
    |a|e^{i(\theta-\phi)} +  |b|e^{i(\chi-\phi')} = 0.\label{second}
\end{eqnarray}
Taking eq. \ref{second} we can infer
\begin{equation*}
\begin{aligned}
    |a|e^{i(\theta-\phi)} &=-  |b|e^{i(\chi-\phi')} ,\\
    &= (-1)|b|e^{i(\chi-\phi')} ,\\
    &= (e^{-i\pi})|b|e^{i(\chi-\phi')} ,\\
    &= |b|e^{i(\chi-\phi'-\pi)}.
    \end{aligned}
\end{equation*}
This results in two conditions 
\begin{eqnarray}
    |a| = |b|, \label{cond1}\\
     \theta-\phi = \chi-\phi'-\pi. \label{cond2}
\end{eqnarray}
Using eq. \ref{cond1} in \ref{first} leads to
\begin{equation*}
\begin{aligned}
    1= |a|e^{i(\theta+\phi)} + |b|e^{i(\chi+\phi')} &= |a|e^{i(\theta+\phi)} + |a|e^{i(\chi+\phi')},\\
    & = |a|\left(e^{i(\theta+\phi)} + e^{i(\chi+\phi')}\right),
\end{aligned}
\end{equation*}
therefore
\begin{equation}
    \frac{1}{|a|} = e^{i(\theta+\phi)} + e^{i(\chi+\phi')}.\label{modExp}
\end{equation}
Last equation means that the right hand side must be purely real, since the left hand side is. As a consequence we can conclude that the imaginary part of $e^{i(\theta+\phi)}$ must be minus the imaginary part of $e^{i(\chi+\phi')}$ in order for them to cancel each other. This happens when either of the following two conditions is met 
\begin{equation}
    \theta+\phi = - (\chi+\phi'), \quad \text{ or } \quad\theta+\phi = \chi+\phi' + \pi \label{38}
\end{equation}
we can check this graphically in Fig. \ref{circle2}, or analitically using Euler's identity $e^{i\phi} = \cos(\phi) + i\sin(\phi)$
\begin{equation*}
\begin{aligned}
    e^{-i\phi} &= \cos(-\phi) + i\sin(-\phi),\\
     &= \cos(\phi) - i\sin(\phi).\\
     e^{i(\phi+\pi)} &= \cos(\phi+\pi) + i\sin(\phi+\pi),\\
     &= \cos(\phi)\cos(\pi) - \sin(\phi)\sin(\pi) + i(\sin(\phi)\cos(\pi) + \sin(\pi)\cos(\phi)),\\
     &= -\cos(\phi) - i\sin(\phi).
\end{aligned}
\end{equation*}
As we can see the second equation in \ref{38} would lead to zero in the right hand side of eq. \ref{modExp}. Hence we will discard that equation and use the other one. Our next step will be adding eqs. \ref{38} and \ref{cond2}
\begin{equation}
    \begin{aligned}[b]
        (\theta+\phi) + (\theta-\phi)&= - (\chi+\phi') + (\chi-\phi'-\pi),\\
        \Rightarrow 2\theta &= -2\phi' - \pi,\\
        \Rightarrow \theta &= -\phi' - \pi/2.
    \end{aligned}
    \label{42}
\end{equation}
If instead of adding them we subtract them the result will be 
\begin{equation}
    \begin{aligned}[b]
        (\theta+\phi) - (\theta-\phi)&= - (\chi+\phi') - (\chi-\phi'-\pi),\\
        \Rightarrow 2\phi &= -2\chi + \pi,\\
        \Rightarrow \chi &= -\phi - \pi/2.
    \end{aligned}
    \label{43}
\end{equation}
If we now go back to the first equation in \ref{38} and plug in eqs. \ref{42} and \ref{43} we will be able to find a relationship between the phase factors $\phi$ and $\phi'$
\begin{equation}
    \begin{aligned}[b]
        \theta+\phi &= - (\chi+\phi'),\\
        \Rightarrow -\phi' - \pi/2 +\phi &= -(-\phi - \pi/2)+\phi',\\
        -2\phi' +2\phi &= \pi,\\
        \phi - \phi' = \pi/2
    \end{aligned}
\end{equation}
In other words, if we want to recover the fermionic creation and annihilation operators $c_k$, $c_k^\dagger$ we have to choose a pair of Majorana operators $\gamma_k$, $\gamma_{k'}$ such that they have a phase difference of $\pi/2$. The simplest choice is $\phi=0$ and $\phi'=-\pi/2$, in this case the resulting operators are 
\begin{eqnarray}
    \gamma_k = e^{i0}c_k + e^{-i0}c_k^\dagger = c_k + c_k^\dagger,\nonumber \\
    \gamma_k = e^{-i\pi/2}c_k + e^{-i(-\pi/2)}c_k^\dagger = -ic_k +i c_k^\dagger = i(c_k^\dagger - c_k)\nonumber,
\end{eqnarray}
Just out of convention, and since both operators are associated with a single fermionic site, we label them as follows
\begin{eqnarray}
    \gamma_{2k-1} = c_k + c_k^\dagger,\label{def1}\\
    \gamma_{2k} = i(c_k^\dagger - c_k)\label{def2}.
\end{eqnarray}
From eqs. \ref{def1} and \ref{def2} we can now easily find how to express $c_k$ and $c_k^\dagger$
\begin{eqnarray}
    c_k = \frac{\gamma_{2k-1} + i \gamma_{2k}}{2},\label{cMaj}\\
    c_k^\dagger = \frac{\gamma_{2k-1} - i \gamma_{2k}}{2} \label{cMaj2}
\end{eqnarray}
We can then, finally, write the number operator
\begin{equation}
    \begin{aligned}[b]
        c_k^\dagger c_k &=  \left(\frac{\gamma_{2k-1} - i \gamma_{2k}}{2}\right) \left(\frac{\gamma_{2k-1} + i \gamma_{2k}}{2}\right),\\
        & = \frac{(\gamma_{2k-1} - i \gamma_{2k})(\gamma_{2k-1} + i \gamma_{2k})}{2},\\
        & = \frac{\gamma_{2k-1}\gamma_{2k-1} - i \gamma_{2k}\gamma_{2k-1} + i \gamma_{2k-1}\gamma_{2k} +  \gamma_{2k}\gamma_{2k}}{2},\\
        \text{using } \gamma_k^\dagger = \gamma_k \quad &= \frac{\gamma_{2k-1}^\dagger\gamma_{2k-1} - i \gamma_{2k}\gamma_{2k-1} + i \gamma_{2k-1}\gamma_{2k} +  \gamma_{2k}^\dagger\gamma_{2k}}{2},\\
        \text{using } \{\gamma_k,\gamma_l\} = 2\delta_{kl} \quad &= \frac{\gamma_{2k-1}^\dagger\gamma_{2k-1} - i \gamma_{2k}\gamma_{2k-1} - i \gamma_{2k}\gamma_{2k-1} +  \gamma_{2k}^\dagger\gamma_{2k}}{2},\\
        \text{using eq. \ref{quadOne}} \quad &= \frac{1 - 2i \gamma_{2k}\gamma_{2k-1} +  1}{2},\\
        &= \frac{2 - 2i \gamma_{2k}\gamma_{2k-1}}{2},\\
        &= 1 - i\gamma_{2k}\gamma_{2k-1}.
    \end{aligned}
    \label{numMaj}
\end{equation}

By now we should be a little familiar with the mathematics and the basic operators we need, but the physical meaning has been relegated. So it is a good time to highlight important concepts. In eqs. \ref{def1} and \ref{def2} its showed that a Majorana particle is born as a linear combination of creation and annihilation operators, this is, it consists of particles and holes in the same proportion. Also, as eqs. \ref{cMaj} and \ref{cMaj2} reveal, each fermion site has two Majorana particles associated to it. This might sound like a peculiarity, but we will see later that this are indeed fundamental properties.\\

Finally, we would like to discuss a little bit of what happens to our Majorana operators when we have gauge symmetry. In other words, what happens to $\gamma_{2k-1}$, and $\gamma_{2k}$ if $c_k$ and $c_k^\dagger$ satisfy
\begin{eqnarray}
    c_k \rightarrow e^{i\phi}c_k,\nonumber\\
    c_k^\dagger \rightarrow e^{-i\phi}c_k^\dagger.\nonumber
\end{eqnarray}
Let's check.
\begin{equation}
    \begin{aligned}[b]
        \gamma_{2k-1} = c_k + c_k^\dagger &\rightarrow e^{i\phi}c_k + e^{-i\phi}c_k^\dagger,\\
        &= (\cos(\phi) + i\sin(\phi))c_k + (\cos(-\phi) + i \sin(-\phi))c_k^\dagger,\\
        &= \cos(\phi)c_k + i\sin(\phi)c_k + \cos(\phi)c_k^\dagger - i \sin(\phi)c_k^\dagger,\\
        &= \cos(\phi)(c_k + c_k^\dagger) +\sin(\phi)(i(c_k -c_k^\dagger)),\\
        &= \cos(\phi)(c_k + c_k^\dagger) -\sin(\phi)(i(c_k^\dagger -c_k)),\\
        &= \cos(\phi)\gamma_{2k-1} -\sin(\phi)\gamma_{2k}.\label{mix1}
    \end{aligned}
\end{equation}
\begin{equation}
    \begin{aligned}[b]
        \gamma_{2k} = i(c_k^\dagger - c_k) &\rightarrow i(e^{-i\phi}c_k^\dagger - e^{i\phi}c_k),\\
        &= i((\cos(\phi) - i\sin(\phi))c_k^\dagger - (\cos(\phi) + i\sin(\phi))c_k),\\
        &= i\cos(\phi)c_k^\dagger  + \sin(\phi)c_k^\dagger - i\cos(\phi)c_k + \sin(\phi)c_k,\\
        &= \cos(\phi)(i(c_k^\dagger-c_k)) + \sin(\phi)(c_k + c_k^\dagger),\\\
        &= \cos(\phi)\gamma_{2k} + \sin(\phi)\gamma_{2k-1}.\label{mix2}
    \end{aligned}
\end{equation}
Physically this means that our modes are getting mixed, as a global phase is introduced. In contrast to the indifference fermions show against this symmetry, for Majorana modes there are physical consequences. Two special instances can be highlighted where this mixing can be avoided: by choosing $\phi=0$ or $\phi=\pi$.\footnote{This is a partial true, because $\pi/2$ and $3\pi/2$ wouldn't mix them, in these cases one operator is transformed into the other, one question then would be why do they ignore these cases usually when they want to break the symmetry?}\label{quest2} Naturally this is a property we would like to have, an interesting question we can pose to ourselves is then, is there a system bearing this symmetry? Well, long story short, yes.\\


\newpage
\section{Bogoliuvob-de Gennes equations in tight binding}
This things come from \cite{Zhu2016}

Let's start with the second-quantized Hamiltonian for electrons experiencing an effective two-particle attractive interaction

\begin{equation}
    \mathscr{H} = \int d\bm{r} \psi_\alpha^\dagger(\bm{r})h_\alpha(\bm{r})\psi_\alpha(\bm{r}) - \frac{1}{2} \int \int d\bm{r}d\bm{r}'V_{\text{eff}}(\bm{r},\bm{r}')\psi_\alpha^\dagger(\bm{r})\psi_\beta^\dagger(\bm{r}')\psi_\beta(\bm{r}')\psi_\alpha(\bm{r}).
\end{equation}
Where $\psi_\alpha^\dagger(\bm{r})$ and $\psi_\alpha(\bm{r})$ are creation and annihilation field operators of an electron with spin $\alpha$ at position $\bm{r}$. They obey the anti-commutation relation
\begin{eqnarray}
    \{\psi_\alpha(\bm{r}),\psi_\beta^\dagger(\bm{r}')\} = \delta(\bm{r} - \bm{r}')\delta_{\alpha \beta}.\\
    \{\psi_\alpha(\bm{r}),\psi_\beta(\bm{r}')\} = \{\psi_\alpha^\dagger(\bm{r}),\psi_\beta^\dagger(\bm{r}')\} = 0.\label{2QHamil}
\end{eqnarray}
And the single particle Hamiltonian is given by
\begin{equation}
    h_\alpha(\bm{r}) = \frac{[\frac{\hbar}{i}\nabla_{\bm{r}} +\frac{e}{c}\bm{A}(\bm{r})]^2}{2m_e} -e\phi(\bm{r}) + \alpha\mu_bH(\bm{r}) - E_F.
\end{equation}
where $\bm{A}(\bm{r})$ and $\phi(\bm{r})$ the vector and scalar potentials. $H(\bm{r})$ is the magnetic field, and it is assumed to be in the $z$ direction. The single particle energy is measured with respect to the Fermi energy $E_F$. $V_{\text{eff}}(\bm{r},\bm{r}')$ is chosen to be positive and a prefactor ``$-$'' is introduced to denote the attractive interaction in the second term of eq. \ref{2QHamil}. Also the symmetry relation $V_{\text{eff}}(\bm{r},\bm{r}') = V_{\text{eff}}(\bm{r}',\bm{r})$ holds. Finally we follow Einstein notation, so repeated index implies summation.\\

We generalize the second-quantized Hamiltonian in eq. \ref{2QHamil}, to include the spin-orbit coupling and the spin-flip scattering interactions, in addition to the regular potential scattering. The single-particle part of the Hamiltonian then looks like
\begin{equation}
    H_0 = \int \int d\bm{r}d\bm{r}' \psi_\alpha^\dagger(\bm{r}) h_{\alpha\beta}(\bm{r},\bm{r}') \psi_\beta(\bm{r}')
\end{equation}
where $h_{\alpha\beta}(\bm{r},\bm{r}')$ is general enough to contain the non-local and spin-flip effects. We can then express the field operators in the localized-state basis as
\begin{eqnarray}
    \psi_\alpha(\bm{r}) = \sum_i w(\bm{r} - \bm{R}_i)c_{i\alpha},\\
    \psi_\alpha^\dagger(\bm{r}) = \sum_i w^*(\bm{r} - \bm{R}_i)c_{i\alpha}^\dagger,
\end{eqnarray}
where $c_{i\alpha}^\dagger$ $(c_{i\alpha})$ creates (annihilates) an electron of spin $\alpha$ at site $i$, and $w(\bm{r} - \bm{R}_i)$ is a localized orbital around the atomic site $\R_i$. Some good options for this are atomic orbitals and maximally localized Wannier orbitals.








\subsection{Derivation of BdG Equations in a Tight-Bind Model}

\subsubsection{Local Density of states and Bond Current}

\subsubsection{Optical Conductivity and Superfluid Density in the Lattice Model}

\subsection{Solution to the BdG Equations in the Lattice Model for a Uniform Superconductor}

\subsection{Abrikosov-Gorkov Equations in the Lattice Model}

\newpage
\section{Obtaining Majorana and other boundary modes from the metamorphosis of impurity-induced states:  exact solutions via T-matrix}
This section corresponds to \cite{2018Bena}.\\

The treatment of boundaries is a problem that has been around for a long time in quantum mechanics. There are of course ways to deal with them, among these techniques we can find, diagonalization of tight-binding hamiltonians with open boundary conditions. Analytic solutions involve the enforcing of boundary conditions that make the problems less general. Other options can be using boundary Green's functions and the bulk boundary correspondence.\\

In this paper they propose a different approach, and it works as follow. Instead of considering a finite system with certain boundary types, we need to consider infinite systems with localized impurities that follow the shape of the boundary. The next step is to obtain the impurity-induced states using T-matrix techniques\footnote{I guess I have to check what are these things.}. The final trick is now take the impurity potential to infinity, with this we can obtain the surface states we were looking for.\\

To illustrate this method the paper discusses two different systems dealing with Majorana zero modes: the Kitaev chain, and chiral states in a 2D system described by the spinless Kitaev model\footnote{Once again the system needs to be spinless, otherwise Majorana zero modes can not arise.}. Then the analytical results are compared against numerical tight-binding calculations to show that they match. This technique can be used for other systems supporting both topological and trivial boundary modes\footnote{The paper were they will publish this results is in preparation according to them, I checked arXiv and there's nothing there yet.}, such as models combining s-wave superconductivity, spin-orbit coupling and Zeeman field, Weyl and Dirac materials, topological insulators, and graphene\footnote{A doubt I would have here is how many of this things need to be in the models in order for the method to be successful, can it be just one of these characteristics or do we need a combination of all of them forcefully to use the method?}.\\

The proposed algorith in the paper goes as follows:

\begin{enumerate}
    \item Take an infinite system (1D, 2D, 3D. Doesn't matter)
    \item Introduce a scalar impurity described by a delta function potential
    \item Use the T-matrix formalism to find the eigenenergies and eigenstates of the impurity bound states
    \item Set the impurity potential to infinity to recover the boundary modes
\end{enumerate}

Now on a quick glimpse we could think that point 4 is unnecessary because we already have a delta function there, but the delta is just spatial, and what I mean by this is that in point two we set, for example for 1D, a potential like this $U\delta(x-x_0)$, while in point four we take the limit $U\rightarrow \infty$, this will be further explained.

\subsection{T-Matrix formalism}

A hamiltonian in momentum space will be denoted $\mathcal{H}_p$. We define the unpertubed Matsubara Green's functions as

\begin{equation}
    G_0(\vect{p},i\omega_n) \equiv (i\omega_n-\mathcal{H}_p)^{-1},
\end{equation}

where $\omega_n$ are the Matsubara frequencies. If we have an impurity, the Green function is

\begin{equation}
    G(\vect{p}_1, \vect{p}_2, i\omega_n) = G_0(\vect{p}_1,i\omega_n)\delta(\vect{p}_1 - \vect{p}_2) + G_0(\vect{p}_1, i\omega_n)T(\vect{p}_1,\vect{p}_2,i\omega_n)G_0(\vect{p}_2,i\omega_n),
\end{equation}

where the T-matrix $T(\vect{p}_1,\vect{p}_2,i\omega_n)$ describes the cumulated effect of the impurity-scattering processes. In 1D for the case of an impurity described by a delta function $V_{imp}(x) = V\delta(x)$ the T-matrix is momentum independent and is

\begin{equation}
    T(p_1,p_2,i\omega_n) = [\mathbb{1} - V\cdot \int \frac{dp}{2\pi}G_0(p,i\omega_n)]^{-1}\cdot V,
\end{equation}
in 2D it is as follows
\begin{equation}
    T(p_{1x},p_{1y},p_{2x},p_{2y},i\omega_n) = \delta(p_{1y} - p_{2y})[\mathbb{1} - \int \frac{dp_x}{2\pi} G_0(p_x,p_{1y},i\omega_n)]^{-1}\cdot V.
    \label{kaladz03}
\end{equation}
For the function in 2D we should point out that it is independent on $p_{1x}, p_{2x}$ since the impurity is a delta in $x$, and a delta in $p_{1y} - p_{2y}$ since  the impurity is independent in $y$ position.

The paper assumes zero temperature and uses this formalism to calculate the retarded Green's function obtained through analytical continuation (setting $i\omega_n$ to $E + i\delta$ and $\delta \rightarrow 0^+$).

\subsection{1D Kitaev Chain with Green's functions}

We start with the tight-binding hamiltonian of an infinite spinless Kitaev chain

\begin{equation}
    \mathcal{H}_{TB} = \sum -\mu c_i^\dagger c_i - \left(tc_i^\dagger c_{i+1} - \Delta c_i c_{i+1}^\dagger + h.c. \right),
\end{equation}

where $t$ is the hopping amplitude, $\mu$ the chemical potential, and $\Delta > 0$ the superconducting pairing amplitude. In momentum space this hamiltonian becomes

\begin{equation}
    \mathcal{H}_p^{1D} =
    \begin{pmatrix}
     -\mu/2 - t\cos(p) & i\Delta \sin(p)\\
     -i\Delta \sin(p) & \mu/2 + t\cos(p)
    \end{pmatrix}.
\end{equation}

We then, introduce our delta impurity potential

\begin{equation}
    V_{imp}(x) = U\delta(x)\begin{pmatrix}
    1&0\\
    0&-1
    \end{pmatrix} \equiv U\delta(x)\sigma_z.
    \label{kaladz06}
\end{equation}

The problem of the impurity Yu-Shiba-Rusinov\footnote{No idea what this is.} is solved using the T-matrix formalism. In momentum space the unperturbed retarded Green's function is $\mathcal{G}_0(p,E) = [E+i0-\mathcal{H}_p^{1D}]^{-1}$, the real counterpart is given by the Fourier transform

\begin{equation}
    \mathcal{G}_0(x,E) = \int \frac{dp}{2\pi}\mathcal{G}_0(p,E)e^{ipx}.
\end{equation}

In order to compute the energy of the YSR states as a function of the impurity potential we take $\mu=0$ and we compute analytically the real space Green's function at $x=0$: 

\begin{equation}
    \mathcal{G}_0(0,E) = \begin{pmatrix}
    EX_0(0) & 0\\
    0 & EX_0(0)
    \end{pmatrix},
    \label{kalad07}
\end{equation}
where\\
\colorbox{orange}{05.03.19 I actually tried to solve this and I couldn't get to equation \ref{kalad07} and failed. }\\
\colorbox{orange}{Might be related to the YSR things I don't know.}
\begin{equation}
    X_0(0) = -\frac{1}{\sqrt{t^2 - E^2}}\frac{1}{\sqrt{\Delta^2 - E^2}}.
\end{equation}

Now, if we want the energies of the impurity bound states we need to calculate the poles of the T-matrix

\begin{equation}
    1 \pm U \frac{1}{\sqrt{t^2 - E^2}}\frac{E}{\sqrt{\Delta^2 - E^2}} = 0.
\end{equation}

This equation yields four solutions, two of them lie outside the gap, we are interested in the pair that lie inside the gap

\begin{equation}
    E_\pm = \pm \sqrt{\frac{1}{2}\left[ t^2 + \Delta^2 + U^2 - \sqrt{(t^2 + \Delta^2 + U^2)^2 - 4t^2\Delta^2}\right]}
\end{equation}

When $U\rightarrow0$

\begin{equation*}
    \begin{aligned}
        E_\pm &= \pm \sqrt{\frac{1}{2}\left[ t^2 + \Delta^2 + - \sqrt{(t^2 + \Delta^2)^2 - 4t^2\Delta^2}\right]},\\
        &= \pm \sqrt{\frac{1}{2}\left[ t^2 + \Delta^2 - \sqrt{t^4 + 2t^2\Delta^2 + \Delta^4 - 4t^2\Delta^2}\right]},\\
        &= \pm \sqrt{\frac{1}{2}\left[ t^2 + \Delta^2 - \sqrt{t^4 - 2t^2\Delta^2 + \Delta^4}\right]},\\
        &= \pm \sqrt{\frac{1}{2}\left[ t^2 + \Delta^2 - \sqrt{(t^2 - \Delta^2)^2}\right]},\\
        &= \pm \sqrt{\frac{1}{2}\left[ t^2 + \Delta^2 - (t^2 - \Delta^2)\right]},\\
        &= \pm \sqrt{\frac{1}{2}\left[ t^2 + \Delta^2 - t^2 + \Delta^2)\right]},\\
        &= \pm \sqrt{\frac{1}{2}\left[ 2\Delta^2\right]},\\
        &= \pm \sqrt{\Delta^2},\\
        &= \pm \Delta
    \end{aligned}
\end{equation*}

whereas when $U\rightarrow \infty$ the solutions decay as 

\begin{equation}
    E_\pm = \pm \frac{\Delta}{U/t}. 
\end{equation}

If we have an infinite potential then, the energies of the bound states decay to zero, i.e. $E_\pm \rightarrow 0$. In the remaining part of the paper $x$ is to be considered a multiple of the lattice parameter 
$a$ (here assumed as 1), in other words $x=na$, $n\in \mathbb{Z}$. This allows us to obtain an exact form for both bounded wavefunctions

\begin{eqnarray}
    \Psi_1(x) \propto \begin{pmatrix} 1 \\ -sgn(x) \end{pmatrix} e^{-\frac{1}{2}\ln \left(\frac{1+\Delta/t}{1-\Delta/t}\right)|x|}\sin\left(\frac{\pi|x|}{2}\right),\\
    \Psi_2(x) \propto \begin{pmatrix} -sgn(x) \\ 1 \end{pmatrix} e^{-\frac{1}{2}\ln \left(\frac{1+\Delta/t}{1-\Delta/t}\right)|x|}\sin\left(\frac{\pi|x|}{2}\right).
\end{eqnarray}

We note that by combining both states we can create the Majorana zero modes. A simple example can be used to illustrate this point, let's forget for a while about the normalization and add both, we will get 

\begin{equation*}
    \Psi_- = \begin{pmatrix} 1 -sgn(x)\\ -sgn(x) + 1 \end{pmatrix} e^{-\frac{1}{2}\ln \left(\frac{1+\Delta/t}{1-\Delta/t}\right)|x|}\sin\left(\frac{\pi|x|}{2}\right)
\end{equation*}

It's clear that the function will be localized at the left of $x=0$ and will vanish at the right side. And mutatis mutandi for the next function
that will vanish at the left of $x=0$
\begin{equation*}
    \Psi_- = \begin{pmatrix} 1 + sgn(x)\\ -sgn(x) - 1 \end{pmatrix} e^{-\frac{1}{2}\ln \left(\frac{1+\Delta/t}{1-\Delta/t}\right)|x|}\sin\left(\frac{\pi|x|}{2}\right)
\end{equation*}

In the paper this results are compared against the numerical ones derived from diagonalizing the 1D Kitaev chain with an impurity\footnote{I wonder how one does that though.}, both match.

\subsection{2D Kitaev model}

As in the previous section our starting point is the hamiltonian of this system

\begin{equation}
    \mathcal{H}_{TB}^{2D} = \sum_{m,n} - \mu c^\dagger_{m,n} c_{m,n} - \left[t\left(c^\dagger_{m+1,n}c_{m,n} + c^\dagger_{m,n+1}c_{m,n}\right) - \Delta(c_{m,n}c_{m+1,n} - i c_{m,n}c_{m,n+1}) + H.c. \right],
\end{equation}

where again $\mu$ is the chemical potential, $t$ the hopping parameter, and $\Delta > 0$ the pairing amplitude. The hamiltonian in momentum space is given by

\begin{equation}
    \mathcal{H}_{\vect{p}}^{2D} = \begin{pmatrix} \epsilon_{\vect{p}} & \Delta_{\vect{p}} \\ \Delta_{\vect{p}}^* & - \epsilon_{\vect{p}}
    \end{pmatrix},
\end{equation}

where $\epsilon_{\vect{p}}= -\mu/2 - t(\cos(p_x) + \cos(p_y))$, $\Delta_{\vect{p}} = i\Delta(\sin(p_x) + i\sin(p_y))$.

The impurity we want (along the z axis) can be described using eq. \ref{kaladz06}. According to eq. \ref{kaladz03} the poles of the T-matrix (which correspond to the impurity energy levels) depend on $p_y$and can be obtained by solving

\begin{equation}
    \det \left[ \mathbb{1}_2 - U\sigma_z \cdot \int \frac{dp_x}{2\pi}\mathcal{G}_0(p_x,p_{1y},E)\right] =0.
    \label{kaladz16}
\end{equation}

At low energies the hamiltonian can be approximated as

\begin{equation}
    \mathcal{H}_{\vect{p}}^{2D} \approx \begin{pmatrix} \xi_{\vect{p}} & i\varkappa (p_x + ip_y) \\ -i\varkappa (p_x + ip_y) &  -\xi_{\vect{p}} \end{pmatrix},
\end{equation}
with $\xi_{\vect{p}} = \frac{\vect{p}^2}{2m_0} - \frac{\vect{p}_F^2}{2m_0}$, where $\vect{p}_F$ is the Fermi momentum, $m_0$ is the quasiparticle mass, and $\varkappa$ the p-wave pairing parameter. For this low energy approximation we can obtain an analytical solution for eq. \ref{kaladz16} for the poles of the T-matrix. The final result is 

\colorbox{yellow}{06.03.19 It would be a good idea to check this result.}
\begin{equation}
    E_\pm = \pm \varkappa p_y.
\end{equation}

In order to get the Majorana modes we take the limit $p_y\rightarrow 0$ and hence $E\rightarrow0$. Those two solutions correspond to counterpropagating chiral Majorana modes.\\

The results are finally compared. The average perturbed spectral function($A(\vect{p},E) = -\frac{1}{\pi}\Im\{\text{Tr}[\mathcal{G}(\vect{p},\vect{p},y)] \}$) is plotted in order to get the energy spectrum. The poles of this function contain both, the unperturbed and the impurity-induced band structures. The analytical and numerical results agree in the vicinity of $p_y=0$ (according to our approximation when $p_y\rightarrow0$).\\

\colorbox{orange}{06.03.19 In principle this is finished, but it has a lot of textual phrases from the paper}\\
\colorbox{orange}{so not all of it is mine. Either rewrite those parts or cite correctly.}

\newpage
\section{Presence versus absence of end-to-end nonlocal conductance correlations in Majorana nanowires: Majorana bound states versus Andreev bound states}
This section corresponds to \cite{2019Lai}.\\

Around the time Kitaev published his paper about Majorana zero modes at the end of a 1D chain, it was found that such zero modes in a topological p-wave superconductor can be detected by measuring the normal metal-superconductor (NS) differential tunneling conductance. If the Majoranas are there this would be reflected in a zero bias conductance peak (ZBCP). For a while this was considered as one of the smoking guns for the experimental search of such states. In fact this result was also discovered and expanded by other groups as well. So it seemed to be the case that the presence or absence of a ZBCP in the NS tunneling indicated the existence or absence of Majorana bound states. This lead to a huge development of materials that could display this property, and in fact a lot of them were discovered. Even surpassing the quality of the originally reported ones by far. If this is the case then, an obvious question would be, how come there is still debate about if Majorana zero modes have been observed or not?

\colorbox{orange}{14.02 Before the last sentence there must be already a part saying that there is an unsettled debate}

\colorbox{orange}{about the experimental observation of majorana zero modes.}

In 2017, the question was raised over whether all the ZBCP observed so far could have been originated from another sources, like Andreev bound states (ABS) which are close to zero energy, and inside the supercoducting gap \cite{PhysRevB.96.075161}. The posibility of having near-zero-energy states generating the ZBCP was not new and the fact that it could arise from ABS was proved experimentally. Finally we are left with the problem that we can not assume  a majorana zero mode is there just because a ZBCP is observed experimentally. Another important fact to point out is that. The presence of low-energy midgap Andreev bound states seems to be rather generic in nanowires like the one in the original paper of Kitaev. The collaboration between Zeeman splitting and spin orbit coupling (SOC) allows this ABS to lie in such low energies. A key difference between ABS and majorana zero modes is that the later arise as zero energy modes only in the topological regime (when the Zeeman spin splitting is larger than the critical field neccessary for a topological quantum phase transition (TQPT)), whereas the former arise in the trivial regime (Zeeman field below critical field). This solves a problem but adds another one, there is no way to experimentally precisely know the critical field.\\

In this paper a way to discern between one or the other, is by careful comparison between the conductance spectra while doing tunneling measurements from the two ends of the nanowire.\\

The setup is very simple, there are leads at both ends, connected by a semiconductor nanowire, and extra details are added on top of it. It goes as follows, from left to right: lead potential barrier, quantum dot, s-wave superconductor, potential barrier, lead. It is important to remember that there is still a semiconductor wire connecting both leads. A schematic plot can be found on \cite{PhysRevB.96.075161}. The conductance will be analyzed on both ends (at the leads) $G_\alpha = dI_\alpha / dV_\alpha$, where $\alpha$ is a label for the lead side $L$, or $R$. $I_\alpha$ denotes the current, and $V_\alpha$ the voltage at side $\alpha$. The Hamiltonian for this nanowire is 

\begin{equation}
\begin{aligned}
 \hat{H} = \frac{1}{2} \int dx \hat{\Psi}^\dagger(x) H_{NW} \hat{\Psi}(x),\qquad \qquad \qquad \quad\\
 H_{NW} = \left( -\frac{\hbar^2}{2m^*}\partial_x^2 - i\alpha_R \partial_x \sigma_y - \mu \right)\tau_z + V_z\sigma_x + \Delta (V_z) \tau_x - i\Gamma,
\end{aligned}
\end{equation}
where $\hat{\Psi} = (\hat{\psi}_\uparrow,\hat{\psi}_\downarrow,\hat{\psi}_\downarrow^\dagger,-\hat{\psi}_\uparrow^\dagger)^T$ is the wavefunction in Nambu space. $\sigma_i$ are the pauli matrices in spin space. $\tau_i$ are the Pauli matrices in particle-hole space. $m^*$ is the effective mass of the electrons, $\alpha_R$ the Rashba spin-orbit coupling, $V_z$ a magnetic field induced Zeeman splitting. $\Delta(V_z)$ is the superconducting pairing potential and it is expresed as 
\begin{equation}
    \Delta(V_z) = \Delta_0\sqrt{1-(V_z/V_c)},
\end{equation}
here $\Delta_0$ is the original gap (without the magnetic field), and $V_c$ is the field at which the superconducting gap collapses. Finally $\Gamma$ is a dissipation parameter often observed experimentally. An important comment here is that neither $\Gamma$ nor $V_c$ are essential parts of the theory to distinguish between ABS and Majorana zero modes.

The potential of the previously schematized device is assumed to contain a quantum dot at the left end, that can generate a subgap Andreev bound state in the trivial phase. The hamiltonian of the quantum dot is 
\begin{equation}
    H_{QD} = \left(-\frac{\hbar^2}{2m^*}\partial^2_x - i\alpha_R \partial_x\sigma_y - \mu + V_{dot}(x)\right)\tau_z + V_z\sigma_x - i\Gamma,
\end{equation}

where $V_{dot}(x) = V_D\cos(3\pi x/ 2l_D)$ is the confinement potential in the quantum dot. $l_D$ is the length of the quantum dot and is only a fraction of the nanowire length $L$. The shape of the quantum dot is irrelevant for this analysis. 

The hamiltonians for the leads are

\begin{equation}
    H_{lead} = \left(-\frac{\hbar^2}{2m^*}\partial^2_x - i\alpha_R\partial_x\sigma_y -\mu + E_{lead}\right)\tau_z + V_z\sigma_x - i\Gamma,
\end{equation}

where $E_{lead}$ is added as a gate voltage\footnote{How come is this a gate voltage?}. Each lead induces a barrier at the junction connecting the lead and the nanowire. The hamiltonian for the barrier is given by

\begin{equation}
    H_{barrier} = \left(-\frac{\hbar^2}{2m^*}\partial^2_x - i\alpha_R\partial_x\sigma_y -\mu + V_{barrier}\right)\tau_z + V_z\sigma_x - i\Gamma,
\end{equation}

where $V_{barrier} = E_{barrier}\Pi_{lbarrier}$ is a box-like potential of height $E_{barrier}$ and width $lbarrier$.\\

It is important to notice that these hamiltonians do not overlap in real space, and they are in order, from left to right $H_{lead}$, $H_{barrier}$, $H_{QD}$, $H_{NW}$, $H_{barrier}$, and finally $H_{lead}$. Temperature effects are taken into consideration. The finite-temperature conductance ($G_T$) is obtained with 
\begin{equation}
    G_T(V) = -\int^\infty_{-\infty} dEG_0(E) \frac{df(E-V)}{dE}.
\end{equation}
where $f(E)$ is the Fermi-Dirac distribution.\\

The temperature conductance $G_0=dI/dV$ is calculated by discretizing the hamiltonians into a lattice chain and obtaining the scattering matrix\footnote{I have no idea how to do this.} using a python package called Kwant. The zero-temperature conductance in units of $e^2/h$ is computed using

\begin{equation}
    G_0 = 2 + \sum_{\sigma,\sigma' = \uparrow, \downarrow} \left(|r_{eh}^{\sigma \sigma'}|^2 - |r_{ee}^{\sigma \sigma'}|^2\right),
\end{equation}

where $r_{eh}$ and $r_{ee}$ are the Andreev, and normal reflection respectively. The 2 in the same equation comes from the contribution of two spin channels while we consider a one-subband system.

In the tunneling limit\footnote{What is that?} conductance can be understood from the local density of states. Such density can be estimated from the energy spectrum and the wavefunction density $|Psi(x)|^2$. To calculate this the barriers are ignored\footnote{Why?}, so the total hamiltonian is

\begin{equation}
    \begin{aligned}
        H_{tot} = H_{QD} + H_{NW} + H_{t},\\
        H_t = u + u^\dagger, \hat{f}_\alpha^\dagger\left(-t\delta_{\alpha\beta} + i\alpha_R \sigma_{\alpha\beta}^y\right)\hat{c}_\beta + h.c.,
    \end{aligned}
\end{equation}

here $H_t$ is the coupling between the quantum dot and the nanowire. $\hat{f}$ creates an electron at the end of the dot, adjacent to the nanowire. $\hat{c}$ annihilates an electron at the end of the nanowire adjacent to the dot. The difference is subtle but it consists of where is the electron placed, in both sides is at the junction of the dot and the nanowire, but in one case is in the nanowire side, and in the other case is at the quantum dot side.\\

The idea of the paper is to see if the ZBCPs at both ends show correlation when the majorana bound states or the ABS arise. The energy spectrum and the form of the lowest lying wave functions is also analyzed. We can take the eigenfunctions $\phi_\epsilon(n) = (u_{n\uparrow},u_{n\downarrow},v_{n\uparrow},v_{n\downarrow})^T$ and their negative energy counterpart $\phi_{-\epsilon}(n) = (v_{n\uparrow}^*,v_{n\downarrow}^*,u_{n\uparrow}^*,u_{n\downarrow}^*)^T$ as represented in Nambu space and we combine them so they satisfy the Majorana conditions

\begin{eqnarray}
    \psi_A(n) =\frac{1}{\sqrt{2}}[\phi_\epsilon(n) + \phi_{-\epsilon}(n)],\\
    \psi_B(n) =-\frac{i}{\sqrt{2}}[\phi_\epsilon(n) - \phi_{-\epsilon}(n)].
\end{eqnarray}

In general $\psi_A(n)$ and $\psi_B(n)$ are not eigenstates of the BdG Hamiltonian, except when $\epsilon = 0$. We can proceed to check the form of both wavefunctions. If $|\psi_A(n)|^2$ and $|\psi_B(n)|^2$ are localized at opposite ends of the nanowire then they are a majorana pair, which means the ZBCP we see in the conductance plots results from majorana zero modes. On the other hand, if we cannot clearly separate both states then the ZBCP arised from ABS.\\


\colorbox{green}{27.03.2019 That's the idea in principle. The rest of the paper is results and discussion.}


\newpage
\section{Distinguishing Majorana Zero Modes from Impurity States through Time-Resolved Transport}
This section corresponds to \cite{tuovinen}.\\



\newpage


\section{Questions}
\begin{enumerate}
    \item Does somebody has any idea why the anticommutation relations should be like this \ref{qs1}. They make sense after we define the $\gamma_k$ but do they make sense before?
    \item Check the comment on this \ref{quest2} and the footnote, the question is just out of curiosity.
\end{enumerate}

\newpage
\section{To do}
\begin{itemize}
    \item Numerically obtain the phase diagram of the phases discussed in \cite{2001kitaev}
\end{itemize}

\subsection{Further reading}

Here are some papers (with their abstracts) that might be the next in line to read. \\

More interesting references can also be found on this page \url{http://iopscience.iop.org/journal/1367-2630/page/Focus\%20on\%20Majorana\%20Fermions\%20in\%20Condensed\%20Matter}


\subsubsection{Majorana Fermions in Particle Physics, Solid
State and Quantum Information}

Corresponds to this reference \cite{2016Borsten}.\\

This review is based on lectures given by M. J. Duff summarising the far reaching contributions of Ettore Majorana to fundamental physics, with special focus on Majorana fermions in all their guises. The theoretical discovery of the eponymous fermion in 1937 has since had profound implications for particle physics, solid state and quantum computation. The breadth of these disciplines is testimony to Majorana's genius, which continues to permeate physics today. These lectures offer a whistle-stop tour through some limited subset of the key ideas. In addition to touching on these various applications, we will draw out some fascinating relations connecting the normed division algebras $\mathbb{R},\mathbb{C},\mathbb{H},\mathbb{O}$ to spinors, trialities, K-theory and the classification of stable topological states of symmetry-protected gapped free-fermion systems


\subsubsection{Search for Majorana fermions in superconductors}

Corresponds to this reference \cite{2012Beenakker}.\\

This is a colloquium-style introduction to the midgap excitations in superconductors known as Majorana fermions. These elusive particles, equal to their own antiparticle, may or may not exist in Nature as elementary building blocks, but in condensed matter they can be constructed out of electron and hole excitations. What is needed is a superconductor to hide the charge difference, and a topological (Berry) phase to eliminate the energy difference from zero-point motion. A pair of widely separated Majorana fermions, bound to magnetic or electrostatic defects, has non-Abelian exchange statistics. A qubit encoded in this Majorana pair is expected to have an unusually long coherence time. We discuss strategies to detect Majorana fermions in a topological superconductor, as well as possible applications in a quantum computer. The status of the experimental search is reviewed.\\
Contents:\\
I. What Are They? (Their origin in particle physics; Their emergence in superconductors; Their potential for quantum computing)\\
II. How to Make Them (Shockley mechanism; Chiral p-wave superconductors; Topological insulators; Semiconductor heterostructures)\\
III. How to Detect Them (Half-integer conductance quantization; Nonlocal tunneling; 4$\pi$-periodic Josephson effect; Thermal metal-insulator transition)\\
IV. How to Use Them (Topological qubits; Read out; Braiding)\\
V. Outlook on the Experimental Progress

\subsubsection{Bogoliubov-de Gennes Method and Its Applications. Chapter 2}

Corresponds to this reference \cite{2016zhu}.\\

In this chapter, I give an alternative derivation of the Bogoliubov-
de Gennes equations for superconductors. It is based on a tight-binding model.
A general symmetry of the equations is discussed. A few physically measurable
quantities are derived in terms of the BdG eigenfunctions. The solutions to the BdG equations in the uniform case are provided. Finally, I also make its connection to the lattice Abrikosov-Gorkov equations.

\subsubsection{Quantum Engineering of Majorana Fermions}

Corresponds to this reference \cite{2018Mascot}.\\

Chiral superconductors have the ability to host topologically protected Majorana zero modes which have been proposed as future qubits for topological quantum computing. The recently introduced magnet--superconductor hybrid (MSH) systems consisting of magnetic adatoms deposited on the surface of conventional s-wave superconductors represent a promising candidate for the creation, detection, and manipulation of Majorana modes via scanning tunneling microscopy techniques. Here, we present four examples demonstrating the ability to engineer Majorana fermions in nanoscopic MSH systems by using atomic manipulation techniques to change the system's shape or magnetic structure. This allows for the dimensional tuning of Majorana modes between one and two dimensions, the identification of the system's topological invariant -- the Chern number -- in real space, the creation of chiral Majorana modes of arbitrary length and shape along magnetic domain boundaries, and the creation of a topological switch using magnetic skyrmions. These examples of Majorana fermion engineering can be realised with current experimental techniques and will promise new avenues for the first prototypes of Majorana-based quantum devices.

\subsubsection{Simulating the exchange of Majorana zero modes with a photonic system}

Corresponds to \cite{Xu2016}.\\

The realization of Majorana zero modes is in the centre of intense theoretical and experimental investigations. Unfortunately, their exchange that can reveal their exotic statistics needs manipulations that are still beyond our experimental capabilities. Here we take an alternative approach. Through the Jordan–Wigner transformation, the Kitaev's chain supporting two Majorana zero modes is mapped to the spin-1/2 chain. We experimentally simulated the spin system and its evolution with a photonic quantum simulator. This allows us to probe the geometric phase, which corresponds to the exchange of two Majorana zero modes positioned at the ends of a three-site chain. Finally, we demonstrate the immunity of quantum information encoded in the Majorana zero modes against local errors through the simulator. Our photonic simulator opens the way for the efficient realization and manipulation of Majorana zero modes in complex architectures.

\subsection{Mobius and Majoranas}

\subsubsection{Majorana fermions in a superconducting Mobius strip}

Corresponds to \cite{2016Pang}.\\

\colorbox{red}{06.03.19 I read this paper, they submitted it to nature. It's so badly written that I would consider a}

\colorbox{red}{miracle if it gets published anywhere. Well, they sure have self confidence though.}

\colorbox{red}{It's experimental anyway.}\\

Recently, much attention has been paid to search for Majorana fermions in solid-state systems. Among various proposals there is one based on radio-frequency superconducting quantum interference devices (rf-SQUIDs), in which the appearance of 4$\pi$-period energy-phase relations is regarded as smoking-gun evidence of Majorana fermion states. Here we report the observation of truncated 4$\pi$-period (i.e., 2$\pi$-period but fully skewed) oscillatory patterns of contact resistance on rf-SQUIDs constructed on the surface of three-dimensional topological insulator Bi2Te3. The results reveal the existence of 1/2 fractional modes of Cooper pairs and the occurrence of parity switchings, both of which are necessary signatures accompanied with the formation of Majorana fermion states. 

\subsubsection{Edge Majoranas on locally flat surfaces - the cone and the Möbius band}

Corresponds to \cite{2016Quelle}.\\

In this paper, we investigate the edge Majorana modes in the simplest possible px+ipy superconductor defined on surfaces with different geometry - the annulus, the cylinder, the Möbius band and a cone (by cone we mean a cone with the tip cut away so it is topologically equivalent to the annulus and cylinder)- and with different configuration of magnetic fluxes threading holes in these surfaces. In particular, we shall address two questions: Given that, in the absence of any flux, the ground state on the annulus does not support Majorana modes, while the one on the cylinder does, how is it possible that the conical geometry can interpolate smoothly between the two? Given that in finite geometries edge Majorana modes have to come in pairs, how can a px+ipy state be defined on a Möbius band, which has only one edge? We show that the key to answering these questions is that the ground state depends on the geometry, even though all the surfaces are locally flat. In the case of the truncated cone, there is a non-trivial holonomy, while the non-orientable Möbius band must necessarily support a domain wall. 

\subsubsection{Möbius topological superconductivity in UPt3}

Corresponds to \cite{2017Yanase}.\\

Intensive studies for more than three decades have elucidated multiple superconducting phases and odd-parity Cooper pairs in a heavy fermion superconductor UPt3. We identify a time-reversal invariant superconducting phase of UPt3 as a recently proposed topological nonsymmorphic superconductivity. Combining the band structure of UPt3, order parameter of E2u representation allowed by P63/mmc space group symmetry, and topological classification by K-theory, we demonstrate the nontrivial Z2-invariant of three-dimensional DIII class enriched by glide symmetry. Correspondingly, double Majorana cone surface states appear at the surface Brillouin zone boundary. Furthermore, we show a variety of surface states and clarify the topological protection by crystal symmetry. Majorana arcs corresponding to tunable Weyl points appear in the time-reversal symmetry broken B-phase. Majorana cone protected by mirror Chern number and Majorana flat band by glide-winding number are also revealed. 

\subsubsection{Metallization of a Rashba wire by a superconducting layer in the strong-proximity regime}

\colorbox{green}{14.02 Next article to read}

Corresponds to \cite{PhysRevB.97.165425}.\\

Semiconducting quantum wires defined within two-dimensional electron gases and strongly coupled to thin superconducting layers have been extensively explored in recent experiments as promising platforms to host Majorana bound states. We study numerically such a geometry, consisting of a quasi-one-dimensional wire coupled to a disordered three-dimensional superconducting layer. We find that, in the strong-coupling limit of a sizable proximity-induced superconducting gap, all transverse subbands of the wire are significantly shifted in energy relative to the chemical potential of the wire. For the lowest subband, this band shift is comparable in magnitude to the spacing between quantized levels that arises due to the finite thickness of the superconductor (which typically is $\approx$ 500 meV for a 10-nm-thick layer of aluminum); in higher subbands, the band shift is much larger. Additionally, we show that the width of the system, which is usually much larger than the thickness, and moderate disorder within the superconductor have almost no impact on the induced gap or band shift. We provide a detailed discussion of the ramifications of our results, arguing that a huge band shift and significant renormalization of semiconducting material parameters in the strong-coupling limit make it challenging to realize a topological phase in such a setup, as the strong coupling to the superconductor essentially metallizes the semiconductor. This metallization of the semiconductor can be tested experimentally through the measurement of the band shift.

\subsection{2019 Articles}

\subsubsection{Helical Majorana modes in iron based Dirac superconductors}

Corresponds to this reference \cite{2019Coleman}.\\

We propose that propagating one-dimensional Majorana fermions will develop in the vortex cores of certain iron-based superconductors in the flux phase, most notably Li(Fe1-xCox)As. A key ingredient of our proposal is the presence of bulk 3D Dirac semimetallic touching points, recently observed in ARPES experiments [P. Zhang et al., Nat. Phys. \textbf{15}, 41 (2019)]. Using an effective $\vect{k}\cdot \vect{p}$ model which describes this class of material in the vicinity of the $\Gamma-Z$ line, we solve the Bogoliubov-deGennes Hamiltonian in the presence of a vortex, demonstrating the development of gapless one-dimensional helical Majorana modes, protected by $C_4$ symmetry. To expose the topological origin of these modes, we use semiclassical methods to evaluate a topological index for arbitrary dispersion beyond the $\vect{k}\cdot \vect{p}$ approximation. This allows us to relate the helical Majorana modes in a vortex line to the presence of monopoles in the Berry curvature of the normal state. We highlight various experimental signatures of our theory and discuss its possible relevance for quantum information applications and the solid state emulation of the early universe

\subsubsection{Finite temperature effects on Majorana bound states in chiral p-wave superconductors}

Corresponds to \cite{2019Schou}.\\

We study Majorana fermions bound to vortex cores in a chiral p-wave superconductor at temperatures non-negligible compared to the superconducting gap. Thermal occupation of Caroli de Gennes-Matricon states, below the full gap, causes the free energy difference between the two fermionic parity sectors to decay algebraically with increasing temperature. The power law acquires an additional factor of T-1 for each bound state thermally excited. The zero-temperature result is exponentially recovered well below the minigap (lowest-lying CdGM level). Our results suggest that temperatures larger than the minigap may not be disastrous for topological quantum computation. 

\subsubsection{Anomalous Nonlocal Conductance as a Fingerprint of Chiral Majorana Edge States}

Corresponds to \cite{2019Ikegaya}.\\

Chiral p-wave superconductor is the primary example of topological systems hosting chiral Majorana edge states. Although candidate materials exist, the conclusive signature of chiral Majorana edge states has not yet been observed in experiments. Here we propose a smoking-gun experiment to detect the chiral Majorana edge states on the basis of theoretical results for the nonlocal conductance in a device consisting of a chiral p-wave superconductor and two ferromagnetic leads. The chiral nature of Majorana edge states causes an anomalously long-range and chirality-sensitive nonlocal transport in these junctions. These two drastic features enable us to identify the moving direction of chiral Majorana edge states in the single experimental setup. 

\subsubsection{From spintronics to Majorana bound states}

Corresponds to \cite{2019Zhou}.\\

 Spin-valve structures in which a change of magnetic configuration is responsible for magnetoresistance have enabled impressive advances in spintronics, focusing on magnetically storing and sensing information. However, this mature technology also offers versatile control of the underlying fringing fields and entirely different applications by realizing topologically-nontrivial states. Together with proximity-induced superconductivity in a two-dimensional electron gas, these fringing fields realized in commercially-available spin valves could control Majorana bound states (MBS). Detailed support for the existence and control of MBS is obtained by combining accurate micromagnetic simulation of fringing fields used as an input in Bogoliubov de Gennes equation to calculate low-energy spectrum, wavefunction localization, and local charge neutrality. A generalized condition for quantum phase transition in these structures provides valuable guidance for the MBS evolution and implementing reconfigurable effective topological wires.
 
 \subsubsection{Antiferromagnetism and chiral d-wave superconductivity from an effective t-J-D model for twisted bilayer graphene}
 
 Corresponds to \cite{2019Gu}.\\
 
 Starting from the strong-coupling limit of an extended Hubbard model, we develop a spin-fermion theory to study the insulating phase and pairing symmetry of the superconducting phase in twisted bilayer graphene. Assuming that the insulating phase is an anti-ferromagnetic insulator, we show that fluctuations of the anti-ferromagnetic order in the conducting phase can mediate superconducting pairing. Using a self-consistent mean-field analysis, we find that the pairing wave function has a chiral d-wave symmetry. Consistent with this observation, we show explicitly the existence of chiral Majorana edge modes by diagonalizing our proposed Hamiltonian on a finite-sized system. These results establish twisted bilayer graphene as a promising platform to realize topological superconductivity. 


\newpage

\bibliography{MyNotes.bib}
\bibliographystyle{ieeetr}


%
% \lipsum[3] % Dummy text
%
% \begin{figure}[H] % Example image
% \center{\includegraphics[width=0.5\linewidth]{placeholder}}
% \caption{Example image.}
% \label{fig:speciation}
% \end{figure}
%

%
% %------------------------------------------------
% \newpage
% \subsubsection{Subsubsection 2} % Sub-sub-section
%
% \lipsum[6] % Dummy text
% \begin{wrapfigure}{l}{0.4\textwidth} % Inline image example
%   \begin{center}
%     \includegraphics[width=0.38\textwidth]{fish}
%   \end{center}
%   \caption{Fish}
% \end{wrapfigure}
% \lipsum[7-8] % Dummy text
%


\end{document}
