\begin{itemize}
    \item Numerically obtain the phase diagram of the phases discussed in \cite{2001kitaev}
\end{itemize}

\subsection{Further reading}

Here are some papers (with their abstracts) that might be the next in line to read. \\

More interesting references can also be found on this page \url{http://iopscience.iop.org/journal/1367-2630/page/Focus\%20on\%20Majorana\%20Fermions\%20in\%20Condensed\%20Matter}


\subsubsection{Majorana Fermions in Particle Physics, Solid
State and Quantum Information}

Corresponds to this reference \cite{2016Borsten}.\\

This review is based on lectures given by M. J. Duff summarising the far reaching contributions of Ettore Majorana to fundamental physics, with special focus on Majorana fermions in all their guises. The theoretical discovery of the eponymous fermion in 1937 has since had profound implications for particle physics, solid state and quantum computation. The breadth of these disciplines is testimony to Majorana's genius, which continues to permeate physics today. These lectures offer a whistle-stop tour through some limited subset of the key ideas. In addition to touching on these various applications, we will draw out some fascinating relations connecting the normed division algebras $\mathbb{R},\mathbb{C},\mathbb{H},\mathbb{O}$ to spinors, trialities, K-theory and the classification of stable topological states of symmetry-protected gapped free-fermion systems


\subsubsection{Search for Majorana fermions in superconductors}

Corresponds to this reference \cite{2012Beenakker}.\\

This is a colloquium-style introduction to the midgap excitations in superconductors known as Majorana fermions. These elusive particles, equal to their own antiparticle, may or may not exist in Nature as elementary building blocks, but in condensed matter they can be constructed out of electron and hole excitations. What is needed is a superconductor to hide the charge difference, and a topological (Berry) phase to eliminate the energy difference from zero-point motion. A pair of widely separated Majorana fermions, bound to magnetic or electrostatic defects, has non-Abelian exchange statistics. A qubit encoded in this Majorana pair is expected to have an unusually long coherence time. We discuss strategies to detect Majorana fermions in a topological superconductor, as well as possible applications in a quantum computer. The status of the experimental search is reviewed.\\
Contents:\\
I. What Are They? (Their origin in particle physics; Their emergence in superconductors; Their potential for quantum computing)\\
II. How to Make Them (Shockley mechanism; Chiral p-wave superconductors; Topological insulators; Semiconductor heterostructures)\\
III. How to Detect Them (Half-integer conductance quantization; Nonlocal tunneling; 4$\pi$-periodic Josephson effect; Thermal metal-insulator transition)\\
IV. How to Use Them (Topological qubits; Read out; Braiding)\\
V. Outlook on the Experimental Progress

\subsubsection{Bogoliubov-de Gennes Method and Its Applications. Chapter 2}

Corresponds to this reference \cite{2016zhu}.\\

In this chapter, I give an alternative derivation of the Bogoliubov-
de Gennes equations for superconductors. It is based on a tight-binding model.
A general symmetry of the equations is discussed. A few physically measurable
quantities are derived in terms of the BdG eigenfunctions. The solutions to the BdG equations in the uniform case are provided. Finally, I also make its connection to the lattice Abrikosov-Gorkov equations.

\subsubsection{Quantum Engineering of Majorana Fermions}

Corresponds to this reference \cite{2018Mascot}.\\

Chiral superconductors have the ability to host topologically protected Majorana zero modes which have been proposed as future qubits for topological quantum computing. The recently introduced magnet--superconductor hybrid (MSH) systems consisting of magnetic adatoms deposited on the surface of conventional s-wave superconductors represent a promising candidate for the creation, detection, and manipulation of Majorana modes via scanning tunneling microscopy techniques. Here, we present four examples demonstrating the ability to engineer Majorana fermions in nanoscopic MSH systems by using atomic manipulation techniques to change the system's shape or magnetic structure. This allows for the dimensional tuning of Majorana modes between one and two dimensions, the identification of the system's topological invariant -- the Chern number -- in real space, the creation of chiral Majorana modes of arbitrary length and shape along magnetic domain boundaries, and the creation of a topological switch using magnetic skyrmions. These examples of Majorana fermion engineering can be realised with current experimental techniques and will promise new avenues for the first prototypes of Majorana-based quantum devices.

\subsubsection{Simulating the exchange of Majorana zero modes with a photonic system}

Corresponds to \cite{Xu2016}.\\

The realization of Majorana zero modes is in the centre of intense theoretical and experimental investigations. Unfortunately, their exchange that can reveal their exotic statistics needs manipulations that are still beyond our experimental capabilities. Here we take an alternative approach. Through the Jordan–Wigner transformation, the Kitaev's chain supporting two Majorana zero modes is mapped to the spin-1/2 chain. We experimentally simulated the spin system and its evolution with a photonic quantum simulator. This allows us to probe the geometric phase, which corresponds to the exchange of two Majorana zero modes positioned at the ends of a three-site chain. Finally, we demonstrate the immunity of quantum information encoded in the Majorana zero modes against local errors through the simulator. Our photonic simulator opens the way for the efficient realization and manipulation of Majorana zero modes in complex architectures.

\subsection{Mobius and Majoranas}

\subsubsection{Majorana fermions in a superconducting Mobius strip}

Corresponds to \cite{2016Pang}.\\

\colorbox{red}{06.03.19 I read this paper, they submitted it to nature. It's so badly written that I would consider a}

\colorbox{red}{miracle if it gets published anywhere. Well, they sure have self confidence though.}

\colorbox{red}{It's experimental anyway.}\\

Recently, much attention has been paid to search for Majorana fermions in solid-state systems. Among various proposals there is one based on radio-frequency superconducting quantum interference devices (rf-SQUIDs), in which the appearance of 4$\pi$-period energy-phase relations is regarded as smoking-gun evidence of Majorana fermion states. Here we report the observation of truncated 4$\pi$-period (i.e., 2$\pi$-period but fully skewed) oscillatory patterns of contact resistance on rf-SQUIDs constructed on the surface of three-dimensional topological insulator Bi2Te3. The results reveal the existence of 1/2 fractional modes of Cooper pairs and the occurrence of parity switchings, both of which are necessary signatures accompanied with the formation of Majorana fermion states. 

\subsubsection{Edge Majoranas on locally flat surfaces - the cone and the Möbius band}

Corresponds to \cite{2016Quelle}.\\

In this paper, we investigate the edge Majorana modes in the simplest possible px+ipy superconductor defined on surfaces with different geometry - the annulus, the cylinder, the Möbius band and a cone (by cone we mean a cone with the tip cut away so it is topologically equivalent to the annulus and cylinder)- and with different configuration of magnetic fluxes threading holes in these surfaces. In particular, we shall address two questions: Given that, in the absence of any flux, the ground state on the annulus does not support Majorana modes, while the one on the cylinder does, how is it possible that the conical geometry can interpolate smoothly between the two? Given that in finite geometries edge Majorana modes have to come in pairs, how can a px+ipy state be defined on a Möbius band, which has only one edge? We show that the key to answering these questions is that the ground state depends on the geometry, even though all the surfaces are locally flat. In the case of the truncated cone, there is a non-trivial holonomy, while the non-orientable Möbius band must necessarily support a domain wall. 

\subsubsection{Möbius topological superconductivity in UPt3}

Corresponds to \cite{2017Yanase}.\\

Intensive studies for more than three decades have elucidated multiple superconducting phases and odd-parity Cooper pairs in a heavy fermion superconductor UPt3. We identify a time-reversal invariant superconducting phase of UPt3 as a recently proposed topological nonsymmorphic superconductivity. Combining the band structure of UPt3, order parameter of E2u representation allowed by P63/mmc space group symmetry, and topological classification by K-theory, we demonstrate the nontrivial Z2-invariant of three-dimensional DIII class enriched by glide symmetry. Correspondingly, double Majorana cone surface states appear at the surface Brillouin zone boundary. Furthermore, we show a variety of surface states and clarify the topological protection by crystal symmetry. Majorana arcs corresponding to tunable Weyl points appear in the time-reversal symmetry broken B-phase. Majorana cone protected by mirror Chern number and Majorana flat band by glide-winding number are also revealed. 

\subsubsection{Metallization of a Rashba wire by a superconducting layer in the strong-proximity regime}

\colorbox{green}{14.02 Next article to read}

Corresponds to \cite{PhysRevB.97.165425}.\\

Semiconducting quantum wires defined within two-dimensional electron gases and strongly coupled to thin superconducting layers have been extensively explored in recent experiments as promising platforms to host Majorana bound states. We study numerically such a geometry, consisting of a quasi-one-dimensional wire coupled to a disordered three-dimensional superconducting layer. We find that, in the strong-coupling limit of a sizable proximity-induced superconducting gap, all transverse subbands of the wire are significantly shifted in energy relative to the chemical potential of the wire. For the lowest subband, this band shift is comparable in magnitude to the spacing between quantized levels that arises due to the finite thickness of the superconductor (which typically is $\approx$ 500 meV for a 10-nm-thick layer of aluminum); in higher subbands, the band shift is much larger. Additionally, we show that the width of the system, which is usually much larger than the thickness, and moderate disorder within the superconductor have almost no impact on the induced gap or band shift. We provide a detailed discussion of the ramifications of our results, arguing that a huge band shift and significant renormalization of semiconducting material parameters in the strong-coupling limit make it challenging to realize a topological phase in such a setup, as the strong coupling to the superconductor essentially metallizes the semiconductor. This metallization of the semiconductor can be tested experimentally through the measurement of the band shift.

\subsection{2019 Articles}

\subsubsection{Helical Majorana modes in iron based Dirac superconductors}

Corresponds to this reference \cite{2019Coleman}.\\

We propose that propagating one-dimensional Majorana fermions will develop in the vortex cores of certain iron-based superconductors in the flux phase, most notably Li(Fe1-xCox)As. A key ingredient of our proposal is the presence of bulk 3D Dirac semimetallic touching points, recently observed in ARPES experiments [P. Zhang et al., Nat. Phys. \textbf{15}, 41 (2019)]. Using an effective $\vect{k}\cdot \vect{p}$ model which describes this class of material in the vicinity of the $\Gamma-Z$ line, we solve the Bogoliubov-deGennes Hamiltonian in the presence of a vortex, demonstrating the development of gapless one-dimensional helical Majorana modes, protected by $C_4$ symmetry. To expose the topological origin of these modes, we use semiclassical methods to evaluate a topological index for arbitrary dispersion beyond the $\vect{k}\cdot \vect{p}$ approximation. This allows us to relate the helical Majorana modes in a vortex line to the presence of monopoles in the Berry curvature of the normal state. We highlight various experimental signatures of our theory and discuss its possible relevance for quantum information applications and the solid state emulation of the early universe

\subsubsection{Finite temperature effects on Majorana bound states in chiral p-wave superconductors}

Corresponds to \cite{2019Schou}.\\

We study Majorana fermions bound to vortex cores in a chiral p-wave superconductor at temperatures non-negligible compared to the superconducting gap. Thermal occupation of Caroli de Gennes-Matricon states, below the full gap, causes the free energy difference between the two fermionic parity sectors to decay algebraically with increasing temperature. The power law acquires an additional factor of T-1 for each bound state thermally excited. The zero-temperature result is exponentially recovered well below the minigap (lowest-lying CdGM level). Our results suggest that temperatures larger than the minigap may not be disastrous for topological quantum computation. 

\subsubsection{Anomalous Nonlocal Conductance as a Fingerprint of Chiral Majorana Edge States}

Corresponds to \cite{2019Ikegaya}.\\

Chiral p-wave superconductor is the primary example of topological systems hosting chiral Majorana edge states. Although candidate materials exist, the conclusive signature of chiral Majorana edge states has not yet been observed in experiments. Here we propose a smoking-gun experiment to detect the chiral Majorana edge states on the basis of theoretical results for the nonlocal conductance in a device consisting of a chiral p-wave superconductor and two ferromagnetic leads. The chiral nature of Majorana edge states causes an anomalously long-range and chirality-sensitive nonlocal transport in these junctions. These two drastic features enable us to identify the moving direction of chiral Majorana edge states in the single experimental setup. 

\subsubsection{From spintronics to Majorana bound states}

Corresponds to \cite{2019Zhou}.\\

 Spin-valve structures in which a change of magnetic configuration is responsible for magnetoresistance have enabled impressive advances in spintronics, focusing on magnetically storing and sensing information. However, this mature technology also offers versatile control of the underlying fringing fields and entirely different applications by realizing topologically-nontrivial states. Together with proximity-induced superconductivity in a two-dimensional electron gas, these fringing fields realized in commercially-available spin valves could control Majorana bound states (MBS). Detailed support for the existence and control of MBS is obtained by combining accurate micromagnetic simulation of fringing fields used as an input in Bogoliubov de Gennes equation to calculate low-energy spectrum, wavefunction localization, and local charge neutrality. A generalized condition for quantum phase transition in these structures provides valuable guidance for the MBS evolution and implementing reconfigurable effective topological wires.
 
 \subsubsection{Antiferromagnetism and chiral d-wave superconductivity from an effective t-J-D model for twisted bilayer graphene}
 
 Corresponds to \cite{2019Gu}.\\
 
 Starting from the strong-coupling limit of an extended Hubbard model, we develop a spin-fermion theory to study the insulating phase and pairing symmetry of the superconducting phase in twisted bilayer graphene. Assuming that the insulating phase is an anti-ferromagnetic insulator, we show that fluctuations of the anti-ferromagnetic order in the conducting phase can mediate superconducting pairing. Using a self-consistent mean-field analysis, we find that the pairing wave function has a chiral d-wave symmetry. Consistent with this observation, we show explicitly the existence of chiral Majorana edge modes by diagonalizing our proposed Hamiltonian on a finite-sized system. These results establish twisted bilayer graphene as a promising platform to realize topological superconductivity. 